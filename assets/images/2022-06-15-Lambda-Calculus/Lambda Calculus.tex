\documentclass[11pt]{ctexart}

    \usepackage[breakable]{tcolorbox}
    \usepackage{parskip} % Stop auto-indenting (to mimic markdown behaviour)
    
    \usepackage{iftex}
    \ifPDFTeX
    	\usepackage[T1]{fontenc}
    	\usepackage{mathpazo}
    \else
    	\usepackage{fontspec}
    \fi

    % Basic figure setup, for now with no caption control since it's done
    % automatically by Pandoc (which extracts ![](path) syntax from Markdown).
    \usepackage{graphicx}
    % Maintain compatibility with old templates. Remove in nbconvert 6.0
    \let\Oldincludegraphics\includegraphics
    % Ensure that by default, figures have no caption (until we provide a
    % proper Figure object with a Caption API and a way to capture that
    % in the conversion process - todo).
    \usepackage{caption}
    \DeclareCaptionFormat{nocaption}{}
    \captionsetup{format=nocaption,aboveskip=0pt,belowskip=0pt}

    \usepackage{float}
    \floatplacement{figure}{H} % forces figures to be placed at the correct location
    \usepackage{xcolor} % Allow colors to be defined
    \usepackage{enumerate} % Needed for markdown enumerations to work
    \usepackage{geometry} % Used to adjust the document margins
    \usepackage{amsmath} % Equations
    \usepackage{amssymb} % Equations
    \usepackage{textcomp} % defines textquotesingle
    % Hack from http://tex.stackexchange.com/a/47451/13684:
    \AtBeginDocument{%
        \def\PYZsq{\textquotesingle}% Upright quotes in Pygmentized code
    }
    \usepackage{upquote} % Upright quotes for verbatim code
    \usepackage{eurosym} % defines \euro
    \usepackage[mathletters]{ucs} % Extended unicode (utf-8) support
    \usepackage{fancyvrb} % verbatim replacement that allows latex
    \usepackage{grffile} % extends the file name processing of package graphics 
                         % to support a larger range
    \makeatletter % fix for old versions of grffile with XeLaTeX
    \@ifpackagelater{grffile}{2019/11/01}
    {
      % Do nothing on new versions
    }
    {
      \def\Gread@@xetex#1{%
        \IfFileExists{"\Gin@base".bb}%
        {\Gread@eps{\Gin@base.bb}}%
        {\Gread@@xetex@aux#1}%
      }
    }
    \makeatother
    \usepackage[Export]{adjustbox} % Used to constrain images to a maximum size
    \adjustboxset{max size={0.9\linewidth}{0.9\paperheight}}

    % The hyperref package gives us a pdf with properly built
    % internal navigation ('pdf bookmarks' for the table of contents,
    % internal cross-reference links, web links for URLs, etc.)
    \usepackage{hyperref}
    % The default LaTeX title has an obnoxious amount of whitespace. By default,
    % titling removes some of it. It also provides customization options.
    \usepackage{titling}
    \usepackage{longtable} % longtable support required by pandoc >1.10
    \usepackage{booktabs}  % table support for pandoc > 1.12.2
    \usepackage[inline]{enumitem} % IRkernel/repr support (it uses the enumerate* environment)
    \usepackage[normalem]{ulem} % ulem is needed to support strikethroughs (\sout)
                                % normalem makes italics be italics, not underlines
    \usepackage{mathrsfs}
    

    
    % Colors for the hyperref package
    \definecolor{urlcolor}{rgb}{0,.145,.698}
    \definecolor{linkcolor}{rgb}{.71,0.21,0.01}
    \definecolor{citecolor}{rgb}{.12,.54,.11}

    % ANSI colors
    \definecolor{ansi-black}{HTML}{3E424D}
    \definecolor{ansi-black-intense}{HTML}{282C36}
    \definecolor{ansi-red}{HTML}{E75C58}
    \definecolor{ansi-red-intense}{HTML}{B22B31}
    \definecolor{ansi-green}{HTML}{00A250}
    \definecolor{ansi-green-intense}{HTML}{007427}
    \definecolor{ansi-yellow}{HTML}{DDB62B}
    \definecolor{ansi-yellow-intense}{HTML}{B27D12}
    \definecolor{ansi-blue}{HTML}{208FFB}
    \definecolor{ansi-blue-intense}{HTML}{0065CA}
    \definecolor{ansi-magenta}{HTML}{D160C4}
    \definecolor{ansi-magenta-intense}{HTML}{A03196}
    \definecolor{ansi-cyan}{HTML}{60C6C8}
    \definecolor{ansi-cyan-intense}{HTML}{258F8F}
    \definecolor{ansi-white}{HTML}{C5C1B4}
    \definecolor{ansi-white-intense}{HTML}{A1A6B2}
    \definecolor{ansi-default-inverse-fg}{HTML}{FFFFFF}
    \definecolor{ansi-default-inverse-bg}{HTML}{000000}

    % common color for the border for error outputs.
    \definecolor{outerrorbackground}{HTML}{FFDFDF}

    % commands and environments needed by pandoc snippets
    % extracted from the output of `pandoc -s`
    \providecommand{\tightlist}{%
      \setlength{\itemsep}{0pt}\setlength{\parskip}{0pt}}
    \DefineVerbatimEnvironment{Highlighting}{Verbatim}{commandchars=\\\{\}}
    % Add ',fontsize=\small' for more characters per line
    \newenvironment{Shaded}{}{}
    \newcommand{\KeywordTok}[1]{\textcolor[rgb]{0.00,0.44,0.13}{\textbf{{#1}}}}
    \newcommand{\DataTypeTok}[1]{\textcolor[rgb]{0.56,0.13,0.00}{{#1}}}
    \newcommand{\DecValTok}[1]{\textcolor[rgb]{0.25,0.63,0.44}{{#1}}}
    \newcommand{\BaseNTok}[1]{\textcolor[rgb]{0.25,0.63,0.44}{{#1}}}
    \newcommand{\FloatTok}[1]{\textcolor[rgb]{0.25,0.63,0.44}{{#1}}}
    \newcommand{\CharTok}[1]{\textcolor[rgb]{0.25,0.44,0.63}{{#1}}}
    \newcommand{\StringTok}[1]{\textcolor[rgb]{0.25,0.44,0.63}{{#1}}}
    \newcommand{\CommentTok}[1]{\textcolor[rgb]{0.38,0.63,0.69}{\textit{{#1}}}}
    \newcommand{\OtherTok}[1]{\textcolor[rgb]{0.00,0.44,0.13}{{#1}}}
    \newcommand{\AlertTok}[1]{\textcolor[rgb]{1.00,0.00,0.00}{\textbf{{#1}}}}
    \newcommand{\FunctionTok}[1]{\textcolor[rgb]{0.02,0.16,0.49}{{#1}}}
    \newcommand{\RegionMarkerTok}[1]{{#1}}
    \newcommand{\ErrorTok}[1]{\textcolor[rgb]{1.00,0.00,0.00}{\textbf{{#1}}}}
    \newcommand{\NormalTok}[1]{{#1}}
    
    % Additional commands for more recent versions of Pandoc
    \newcommand{\ConstantTok}[1]{\textcolor[rgb]{0.53,0.00,0.00}{{#1}}}
    \newcommand{\SpecialCharTok}[1]{\textcolor[rgb]{0.25,0.44,0.63}{{#1}}}
    \newcommand{\VerbatimStringTok}[1]{\textcolor[rgb]{0.25,0.44,0.63}{{#1}}}
    \newcommand{\SpecialStringTok}[1]{\textcolor[rgb]{0.73,0.40,0.53}{{#1}}}
    \newcommand{\ImportTok}[1]{{#1}}
    \newcommand{\DocumentationTok}[1]{\textcolor[rgb]{0.73,0.13,0.13}{\textit{{#1}}}}
    \newcommand{\AnnotationTok}[1]{\textcolor[rgb]{0.38,0.63,0.69}{\textbf{\textit{{#1}}}}}
    \newcommand{\CommentVarTok}[1]{\textcolor[rgb]{0.38,0.63,0.69}{\textbf{\textit{{#1}}}}}
    \newcommand{\VariableTok}[1]{\textcolor[rgb]{0.10,0.09,0.49}{{#1}}}
    \newcommand{\ControlFlowTok}[1]{\textcolor[rgb]{0.00,0.44,0.13}{\textbf{{#1}}}}
    \newcommand{\OperatorTok}[1]{\textcolor[rgb]{0.40,0.40,0.40}{{#1}}}
    \newcommand{\BuiltInTok}[1]{{#1}}
    \newcommand{\ExtensionTok}[1]{{#1}}
    \newcommand{\PreprocessorTok}[1]{\textcolor[rgb]{0.74,0.48,0.00}{{#1}}}
    \newcommand{\AttributeTok}[1]{\textcolor[rgb]{0.49,0.56,0.16}{{#1}}}
    \newcommand{\InformationTok}[1]{\textcolor[rgb]{0.38,0.63,0.69}{\textbf{\textit{{#1}}}}}
    \newcommand{\WarningTok}[1]{\textcolor[rgb]{0.38,0.63,0.69}{\textbf{\textit{{#1}}}}}
    
    
    % Define a nice break command that doesn't care if a line doesn't already
    % exist.
    \def\br{\hspace*{\fill} \\* }
    % Math Jax compatibility definitions
    \def\gt{>}
    \def\lt{<}
    \let\Oldtex\TeX
    \let\Oldlatex\LaTeX
    \renewcommand{\TeX}{\textrm{\Oldtex}}
    \renewcommand{\LaTeX}{\textrm{\Oldlatex}}
    % Document parameters
    % Document title
    \title{Lambda Calculus}
    
    
    
    
    
% Pygments definitions
\makeatletter
\def\PY@reset{\let\PY@it=\relax \let\PY@bf=\relax%
    \let\PY@ul=\relax \let\PY@tc=\relax%
    \let\PY@bc=\relax \let\PY@ff=\relax}
\def\PY@tok#1{\csname PY@tok@#1\endcsname}
\def\PY@toks#1+{\ifx\relax#1\empty\else%
    \PY@tok{#1}\expandafter\PY@toks\fi}
\def\PY@do#1{\PY@bc{\PY@tc{\PY@ul{%
    \PY@it{\PY@bf{\PY@ff{#1}}}}}}}
\def\PY#1#2{\PY@reset\PY@toks#1+\relax+\PY@do{#2}}

\@namedef{PY@tok@w}{\def\PY@tc##1{\textcolor[rgb]{0.73,0.73,0.73}{##1}}}
\@namedef{PY@tok@c}{\let\PY@it=\textit\def\PY@tc##1{\textcolor[rgb]{0.25,0.50,0.50}{##1}}}
\@namedef{PY@tok@cp}{\def\PY@tc##1{\textcolor[rgb]{0.74,0.48,0.00}{##1}}}
\@namedef{PY@tok@k}{\let\PY@bf=\textbf\def\PY@tc##1{\textcolor[rgb]{0.00,0.50,0.00}{##1}}}
\@namedef{PY@tok@kp}{\def\PY@tc##1{\textcolor[rgb]{0.00,0.50,0.00}{##1}}}
\@namedef{PY@tok@kt}{\def\PY@tc##1{\textcolor[rgb]{0.69,0.00,0.25}{##1}}}
\@namedef{PY@tok@o}{\def\PY@tc##1{\textcolor[rgb]{0.40,0.40,0.40}{##1}}}
\@namedef{PY@tok@ow}{\let\PY@bf=\textbf\def\PY@tc##1{\textcolor[rgb]{0.67,0.13,1.00}{##1}}}
\@namedef{PY@tok@nb}{\def\PY@tc##1{\textcolor[rgb]{0.00,0.50,0.00}{##1}}}
\@namedef{PY@tok@nf}{\def\PY@tc##1{\textcolor[rgb]{0.00,0.00,1.00}{##1}}}
\@namedef{PY@tok@nc}{\let\PY@bf=\textbf\def\PY@tc##1{\textcolor[rgb]{0.00,0.00,1.00}{##1}}}
\@namedef{PY@tok@nn}{\let\PY@bf=\textbf\def\PY@tc##1{\textcolor[rgb]{0.00,0.00,1.00}{##1}}}
\@namedef{PY@tok@ne}{\let\PY@bf=\textbf\def\PY@tc##1{\textcolor[rgb]{0.82,0.25,0.23}{##1}}}
\@namedef{PY@tok@nv}{\def\PY@tc##1{\textcolor[rgb]{0.10,0.09,0.49}{##1}}}
\@namedef{PY@tok@no}{\def\PY@tc##1{\textcolor[rgb]{0.53,0.00,0.00}{##1}}}
\@namedef{PY@tok@nl}{\def\PY@tc##1{\textcolor[rgb]{0.63,0.63,0.00}{##1}}}
\@namedef{PY@tok@ni}{\let\PY@bf=\textbf\def\PY@tc##1{\textcolor[rgb]{0.60,0.60,0.60}{##1}}}
\@namedef{PY@tok@na}{\def\PY@tc##1{\textcolor[rgb]{0.49,0.56,0.16}{##1}}}
\@namedef{PY@tok@nt}{\let\PY@bf=\textbf\def\PY@tc##1{\textcolor[rgb]{0.00,0.50,0.00}{##1}}}
\@namedef{PY@tok@nd}{\def\PY@tc##1{\textcolor[rgb]{0.67,0.13,1.00}{##1}}}
\@namedef{PY@tok@s}{\def\PY@tc##1{\textcolor[rgb]{0.73,0.13,0.13}{##1}}}
\@namedef{PY@tok@sd}{\let\PY@it=\textit\def\PY@tc##1{\textcolor[rgb]{0.73,0.13,0.13}{##1}}}
\@namedef{PY@tok@si}{\let\PY@bf=\textbf\def\PY@tc##1{\textcolor[rgb]{0.73,0.40,0.53}{##1}}}
\@namedef{PY@tok@se}{\let\PY@bf=\textbf\def\PY@tc##1{\textcolor[rgb]{0.73,0.40,0.13}{##1}}}
\@namedef{PY@tok@sr}{\def\PY@tc##1{\textcolor[rgb]{0.73,0.40,0.53}{##1}}}
\@namedef{PY@tok@ss}{\def\PY@tc##1{\textcolor[rgb]{0.10,0.09,0.49}{##1}}}
\@namedef{PY@tok@sx}{\def\PY@tc##1{\textcolor[rgb]{0.00,0.50,0.00}{##1}}}
\@namedef{PY@tok@m}{\def\PY@tc##1{\textcolor[rgb]{0.40,0.40,0.40}{##1}}}
\@namedef{PY@tok@gh}{\let\PY@bf=\textbf\def\PY@tc##1{\textcolor[rgb]{0.00,0.00,0.50}{##1}}}
\@namedef{PY@tok@gu}{\let\PY@bf=\textbf\def\PY@tc##1{\textcolor[rgb]{0.50,0.00,0.50}{##1}}}
\@namedef{PY@tok@gd}{\def\PY@tc##1{\textcolor[rgb]{0.63,0.00,0.00}{##1}}}
\@namedef{PY@tok@gi}{\def\PY@tc##1{\textcolor[rgb]{0.00,0.63,0.00}{##1}}}
\@namedef{PY@tok@gr}{\def\PY@tc##1{\textcolor[rgb]{1.00,0.00,0.00}{##1}}}
\@namedef{PY@tok@ge}{\let\PY@it=\textit}
\@namedef{PY@tok@gs}{\let\PY@bf=\textbf}
\@namedef{PY@tok@gp}{\let\PY@bf=\textbf\def\PY@tc##1{\textcolor[rgb]{0.00,0.00,0.50}{##1}}}
\@namedef{PY@tok@go}{\def\PY@tc##1{\textcolor[rgb]{0.53,0.53,0.53}{##1}}}
\@namedef{PY@tok@gt}{\def\PY@tc##1{\textcolor[rgb]{0.00,0.27,0.87}{##1}}}
\@namedef{PY@tok@err}{\def\PY@bc##1{{\setlength{\fboxsep}{\string -\fboxrule}\fcolorbox[rgb]{1.00,0.00,0.00}{1,1,1}{\strut ##1}}}}
\@namedef{PY@tok@kc}{\let\PY@bf=\textbf\def\PY@tc##1{\textcolor[rgb]{0.00,0.50,0.00}{##1}}}
\@namedef{PY@tok@kd}{\let\PY@bf=\textbf\def\PY@tc##1{\textcolor[rgb]{0.00,0.50,0.00}{##1}}}
\@namedef{PY@tok@kn}{\let\PY@bf=\textbf\def\PY@tc##1{\textcolor[rgb]{0.00,0.50,0.00}{##1}}}
\@namedef{PY@tok@kr}{\let\PY@bf=\textbf\def\PY@tc##1{\textcolor[rgb]{0.00,0.50,0.00}{##1}}}
\@namedef{PY@tok@bp}{\def\PY@tc##1{\textcolor[rgb]{0.00,0.50,0.00}{##1}}}
\@namedef{PY@tok@fm}{\def\PY@tc##1{\textcolor[rgb]{0.00,0.00,1.00}{##1}}}
\@namedef{PY@tok@vc}{\def\PY@tc##1{\textcolor[rgb]{0.10,0.09,0.49}{##1}}}
\@namedef{PY@tok@vg}{\def\PY@tc##1{\textcolor[rgb]{0.10,0.09,0.49}{##1}}}
\@namedef{PY@tok@vi}{\def\PY@tc##1{\textcolor[rgb]{0.10,0.09,0.49}{##1}}}
\@namedef{PY@tok@vm}{\def\PY@tc##1{\textcolor[rgb]{0.10,0.09,0.49}{##1}}}
\@namedef{PY@tok@sa}{\def\PY@tc##1{\textcolor[rgb]{0.73,0.13,0.13}{##1}}}
\@namedef{PY@tok@sb}{\def\PY@tc##1{\textcolor[rgb]{0.73,0.13,0.13}{##1}}}
\@namedef{PY@tok@sc}{\def\PY@tc##1{\textcolor[rgb]{0.73,0.13,0.13}{##1}}}
\@namedef{PY@tok@dl}{\def\PY@tc##1{\textcolor[rgb]{0.73,0.13,0.13}{##1}}}
\@namedef{PY@tok@s2}{\def\PY@tc##1{\textcolor[rgb]{0.73,0.13,0.13}{##1}}}
\@namedef{PY@tok@sh}{\def\PY@tc##1{\textcolor[rgb]{0.73,0.13,0.13}{##1}}}
\@namedef{PY@tok@s1}{\def\PY@tc##1{\textcolor[rgb]{0.73,0.13,0.13}{##1}}}
\@namedef{PY@tok@mb}{\def\PY@tc##1{\textcolor[rgb]{0.40,0.40,0.40}{##1}}}
\@namedef{PY@tok@mf}{\def\PY@tc##1{\textcolor[rgb]{0.40,0.40,0.40}{##1}}}
\@namedef{PY@tok@mh}{\def\PY@tc##1{\textcolor[rgb]{0.40,0.40,0.40}{##1}}}
\@namedef{PY@tok@mi}{\def\PY@tc##1{\textcolor[rgb]{0.40,0.40,0.40}{##1}}}
\@namedef{PY@tok@il}{\def\PY@tc##1{\textcolor[rgb]{0.40,0.40,0.40}{##1}}}
\@namedef{PY@tok@mo}{\def\PY@tc##1{\textcolor[rgb]{0.40,0.40,0.40}{##1}}}
\@namedef{PY@tok@ch}{\let\PY@it=\textit\def\PY@tc##1{\textcolor[rgb]{0.25,0.50,0.50}{##1}}}
\@namedef{PY@tok@cm}{\let\PY@it=\textit\def\PY@tc##1{\textcolor[rgb]{0.25,0.50,0.50}{##1}}}
\@namedef{PY@tok@cpf}{\let\PY@it=\textit\def\PY@tc##1{\textcolor[rgb]{0.25,0.50,0.50}{##1}}}
\@namedef{PY@tok@c1}{\let\PY@it=\textit\def\PY@tc##1{\textcolor[rgb]{0.25,0.50,0.50}{##1}}}
\@namedef{PY@tok@cs}{\let\PY@it=\textit\def\PY@tc##1{\textcolor[rgb]{0.25,0.50,0.50}{##1}}}

\def\PYZbs{\char`\\}
\def\PYZus{\char`\_}
\def\PYZob{\char`\{}
\def\PYZcb{\char`\}}
\def\PYZca{\char`\^}
\def\PYZam{\char`\&}
\def\PYZlt{\char`\<}
\def\PYZgt{\char`\>}
\def\PYZsh{\char`\#}
\def\PYZpc{\char`\%}
\def\PYZdl{\char`\$}
\def\PYZhy{\char`\-}
\def\PYZsq{\char`\'}
\def\PYZdq{\char`\"}
\def\PYZti{\char`\~}
% for compatibility with earlier versions
\def\PYZat{@}
\def\PYZlb{[}
\def\PYZrb{]}
\makeatother


    % For linebreaks inside Verbatim environment from package fancyvrb. 
    \makeatletter
        \newbox\Wrappedcontinuationbox 
        \newbox\Wrappedvisiblespacebox 
        \newcommand*\Wrappedvisiblespace {\textcolor{red}{\textvisiblespace}} 
        \newcommand*\Wrappedcontinuationsymbol {\textcolor{red}{\llap{\tiny$\m@th\hookrightarrow$}}} 
        \newcommand*\Wrappedcontinuationindent {3ex } 
        \newcommand*\Wrappedafterbreak {\kern\Wrappedcontinuationindent\copy\Wrappedcontinuationbox} 
        % Take advantage of the already applied Pygments mark-up to insert 
        % potential linebreaks for TeX processing. 
        %        {, <, #, %, $, ' and ": go to next line. 
        %        _, }, ^, &, >, - and ~: stay at end of broken line. 
        % Use of \textquotesingle for straight quote. 
        \newcommand*\Wrappedbreaksatspecials {% 
            \def\PYGZus{\discretionary{\char`\_}{\Wrappedafterbreak}{\char`\_}}% 
            \def\PYGZob{\discretionary{}{\Wrappedafterbreak\char`\{}{\char`\{}}% 
            \def\PYGZcb{\discretionary{\char`\}}{\Wrappedafterbreak}{\char`\}}}% 
            \def\PYGZca{\discretionary{\char`\^}{\Wrappedafterbreak}{\char`\^}}% 
            \def\PYGZam{\discretionary{\char`\&}{\Wrappedafterbreak}{\char`\&}}% 
            \def\PYGZlt{\discretionary{}{\Wrappedafterbreak\char`\<}{\char`\<}}% 
            \def\PYGZgt{\discretionary{\char`\>}{\Wrappedafterbreak}{\char`\>}}% 
            \def\PYGZsh{\discretionary{}{\Wrappedafterbreak\char`\#}{\char`\#}}% 
            \def\PYGZpc{\discretionary{}{\Wrappedafterbreak\char`\%}{\char`\%}}% 
            \def\PYGZdl{\discretionary{}{\Wrappedafterbreak\char`\$}{\char`\$}}% 
            \def\PYGZhy{\discretionary{\char`\-}{\Wrappedafterbreak}{\char`\-}}% 
            \def\PYGZsq{\discretionary{}{\Wrappedafterbreak\textquotesingle}{\textquotesingle}}% 
            \def\PYGZdq{\discretionary{}{\Wrappedafterbreak\char`\"}{\char`\"}}% 
            \def\PYGZti{\discretionary{\char`\~}{\Wrappedafterbreak}{\char`\~}}% 
        } 
        % Some characters . , ; ? ! / are not pygmentized. 
        % This macro makes them "active" and they will insert potential linebreaks 
        \newcommand*\Wrappedbreaksatpunct {% 
            \lccode`\~`\.\lowercase{\def~}{\discretionary{\hbox{\char`\.}}{\Wrappedafterbreak}{\hbox{\char`\.}}}% 
            \lccode`\~`\,\lowercase{\def~}{\discretionary{\hbox{\char`\,}}{\Wrappedafterbreak}{\hbox{\char`\,}}}% 
            \lccode`\~`\;\lowercase{\def~}{\discretionary{\hbox{\char`\;}}{\Wrappedafterbreak}{\hbox{\char`\;}}}% 
            \lccode`\~`\:\lowercase{\def~}{\discretionary{\hbox{\char`\:}}{\Wrappedafterbreak}{\hbox{\char`\:}}}% 
            \lccode`\~`\?\lowercase{\def~}{\discretionary{\hbox{\char`\?}}{\Wrappedafterbreak}{\hbox{\char`\?}}}% 
            \lccode`\~`\!\lowercase{\def~}{\discretionary{\hbox{\char`\!}}{\Wrappedafterbreak}{\hbox{\char`\!}}}% 
            \lccode`\~`\/\lowercase{\def~}{\discretionary{\hbox{\char`\/}}{\Wrappedafterbreak}{\hbox{\char`\/}}}% 
            \catcode`\.\active
            \catcode`\,\active 
            \catcode`\;\active
            \catcode`\:\active
            \catcode`\?\active
            \catcode`\!\active
            \catcode`\/\active 
            \lccode`\~`\~ 	
        }
    \makeatother

    \let\OriginalVerbatim=\Verbatim
    \makeatletter
    \renewcommand{\Verbatim}[1][1]{%
        %\parskip\z@skip
        \sbox\Wrappedcontinuationbox {\Wrappedcontinuationsymbol}%
        \sbox\Wrappedvisiblespacebox {\FV@SetupFont\Wrappedvisiblespace}%
        \def\FancyVerbFormatLine ##1{\hsize\linewidth
            \vtop{\raggedright\hyphenpenalty\z@\exhyphenpenalty\z@
                \doublehyphendemerits\z@\finalhyphendemerits\z@
                \strut ##1\strut}%
        }%
        % If the linebreak is at a space, the latter will be displayed as visible
        % space at end of first line, and a continuation symbol starts next line.
        % Stretch/shrink are however usually zero for typewriter font.
        \def\FV@Space {%
            \nobreak\hskip\z@ plus\fontdimen3\font minus\fontdimen4\font
            \discretionary{\copy\Wrappedvisiblespacebox}{\Wrappedafterbreak}
            {\kern\fontdimen2\font}%
        }%
        
        % Allow breaks at special characters using \PYG... macros.
        \Wrappedbreaksatspecials
        % Breaks at punctuation characters . , ; ? ! and / need catcode=\active 	
        \OriginalVerbatim[#1,codes*=\Wrappedbreaksatpunct]%
    }
    \makeatother

    % Exact colors from NB
    \definecolor{incolor}{HTML}{303F9F}
    \definecolor{outcolor}{HTML}{D84315}
    \definecolor{cellborder}{HTML}{CFCFCF}
    \definecolor{cellbackground}{HTML}{F7F7F7}
    
    % prompt
    \makeatletter
    \newcommand{\boxspacing}{\kern\kvtcb@left@rule\kern\kvtcb@boxsep}
    \makeatother
    \newcommand{\prompt}[4]{
        {\ttfamily\llap{{\color{#2}[#3]:\hspace{3pt}#4}}\vspace{-\baselineskip}}
    }
    

    
    % Prevent overflowing lines due to hard-to-break entities
    \sloppy 
    % Setup hyperref package
    \hypersetup{
      breaklinks=true,  % so long urls are correctly broken across lines
      colorlinks=true,
      urlcolor=urlcolor,
      linkcolor=linkcolor,
      citecolor=citecolor,
      }
    % Slightly bigger margins than the latex defaults
    
    \geometry{verbose,tmargin=1in,bmargin=1in,lmargin=1in,rmargin=1in}
    
    

\begin{document}
    
    \maketitle
    
    

    
    \hypertarget{lambda-calculus-for-humans}{%
\section{Lambda Calculus for Humans}\label{lambda-calculus-for-humans}}

\hypertarget{ux524dux8a00}{%
\subsection{前言}\label{ux524dux8a00}}

这是一个写给正常人学习的\(𝜆\)演算教程,如果你搜索「Lambda
演算」,维基百科上的说明长这样:

\begin{verbatim}
0 = λf.λx.x
1 = λf.λx.f x
2 = λf.λx.f (f x)
3 = λf.λx.f (f (f x))
\end{verbatim}

然后你搜索「A Tutorial Introduction to the Lambda
Calculus」,得到的结果更抽象了:

\[
\begin{aligned}
1 \equiv& \lambda s z .s(z) \\
2 \equiv& \lambda s z .s(s(z)) \\
3 \equiv& \lambda s z .s(s(s(z)))\\
\mathrm{S} \equiv& \lambda wyx.y(wyx)\\
\mathrm{S0} \equiv& (\lambda wyx.y(wyx))(\lambda sz.z)
\end{aligned}
\]

研究表明,长时间观看这些抽象的算符会造成心理上的不适,并容易产生智商上的挫败感,这对正常人类的心智是有害的。

因此我肩负着拯救人类的使命,写下了这一份「给正常人看的
\(λ\)-calculus教程」。

\hypertarget{ux51fdux6570ux5f0fux7f16ux7a0b}{%
\subsection{函数式编程}\label{ux51fdux6570ux5f0fux7f16ux7a0b}}

首先,让我们摒弃掉那些\(α\)-conversion、\(β\)-reduction、currying这些让人san值狂掉的术语,先来介绍一个更广为人知的术语「函数式编程」(Functional
Calculus),它的核心思想很简单:\textbf{用函数来表达一切}。

传统编程范式中,我们面向的对象主要是变量,而在函数式编程中,我们面向的对象是函数,举个例子,我们定义一个\texttt{now}函数打印日期

    \begin{tcolorbox}[breakable, size=fbox, boxrule=1pt, pad at break*=1mm,colback=cellbackground, colframe=cellborder]
\prompt{In}{incolor}{1}{\boxspacing}
\begin{Verbatim}[commandchars=\\\{\}]
\PY{k}{def} \PY{n+nf}{now}\PY{p}{(}\PY{n}{data}\PY{p}{)}\PY{p}{:}
    \PY{n+nb}{print}\PY{p}{(}\PY{l+s+s1}{\PYZsq{}}\PY{l+s+s1}{Today is: }\PY{l+s+s1}{\PYZsq{}} \PY{o}{+} \PY{n}{data}\PY{p}{)}

\PY{n}{data} \PY{o}{=} \PY{l+s+s1}{\PYZsq{}}\PY{l+s+s1}{2022\PYZhy{}6\PYZhy{}11}\PY{l+s+s1}{\PYZsq{}}
\PY{n}{now}\PY{p}{(}\PY{n}{data}\PY{p}{)}
\end{Verbatim}
\end{tcolorbox}

    \begin{Verbatim}[commandchars=\\\{\}]
Today is: 2022-6-11
    \end{Verbatim}

    这就是一个非常传统的面向变量的函数:输入一个变量,然后对变量进行操作。

函数式编程输入的就是一个函数,然后对函数进行操作,比如我们定义一个\texttt{log}函数打印输入函数的函数名:

    \begin{tcolorbox}[breakable, size=fbox, boxrule=1pt, pad at break*=1mm,colback=cellbackground, colframe=cellborder]
\prompt{In}{incolor}{2}{\boxspacing}
\begin{Verbatim}[commandchars=\\\{\}]
\PY{k}{def} \PY{n+nf}{log}\PY{p}{(}\PY{n}{func}\PY{p}{)}\PY{p}{:}
    \PY{n+nb}{print}\PY{p}{(}\PY{n}{func}\PY{o}{.}\PY{n+nv+vm}{\PYZus{}\PYZus{}name\PYZus{}\PYZus{}}\PY{p}{)}

\PY{n}{log}\PY{p}{(}\PY{n}{now}\PY{p}{)}
\end{Verbatim}
\end{tcolorbox}

    \begin{Verbatim}[commandchars=\\\{\}]
now
    \end{Verbatim}

    但是这样有一点不好,那就是我们没法在调用\texttt{now}函数的同时也打印出函数名,因此我们将\texttt{log}函数稍微修改一下:

    \begin{tcolorbox}[breakable, size=fbox, boxrule=1pt, pad at break*=1mm,colback=cellbackground, colframe=cellborder]
\prompt{In}{incolor}{3}{\boxspacing}
\begin{Verbatim}[commandchars=\\\{\}]
\PY{k}{def} \PY{n+nf}{log}\PY{p}{(}\PY{n}{func}\PY{p}{)}\PY{p}{:}
    \PY{k}{def} \PY{n+nf}{F}\PY{p}{(}\PY{n}{data}\PY{p}{)}\PY{p}{:}
        \PY{n+nb}{print}\PY{p}{(}\PY{l+s+s1}{\PYZsq{}}\PY{l+s+s1}{call }\PY{l+s+si}{\PYZpc{}s}\PY{l+s+s1}{():}\PY{l+s+s1}{\PYZsq{}} \PY{o}{\PYZpc{}} \PY{n}{func}\PY{o}{.}\PY{n+nv+vm}{\PYZus{}\PYZus{}name\PYZus{}\PYZus{}}\PY{p}{)}
        \PY{k}{return} \PY{n}{func}\PY{p}{(}\PY{n}{data}\PY{p}{)}
    \PY{k}{return} \PY{n}{F}

\PY{n}{log}\PY{p}{(}\PY{n}{now}\PY{p}{)}\PY{p}{(}\PY{n}{data}\PY{p}{)}
\end{Verbatim}
\end{tcolorbox}

    \begin{Verbatim}[commandchars=\\\{\}]
call now():
Today is: 2022-6-11
    \end{Verbatim}

    这样将\texttt{log}函数变成了一个二阶函数,于是一切就OK了。

PS:其实在Python中有一个更优雅的写法,那就是「装饰器」(Decorator),本质就是额外执行了一个
\texttt{now\ =\ log(now)}

    \begin{tcolorbox}[breakable, size=fbox, boxrule=1pt, pad at break*=1mm,colback=cellbackground, colframe=cellborder]
\prompt{In}{incolor}{4}{\boxspacing}
\begin{Verbatim}[commandchars=\\\{\}]
\PY{n+nd}{@log}
\PY{k}{def} \PY{n+nf}{now}\PY{p}{(}\PY{n}{data}\PY{p}{)}\PY{p}{:}
    \PY{n+nb}{print}\PY{p}{(}\PY{l+s+s1}{\PYZsq{}}\PY{l+s+s1}{Today is: }\PY{l+s+s1}{\PYZsq{}} \PY{o}{+} \PY{n}{data}\PY{p}{)}

\PY{n}{now}\PY{p}{(}\PY{n}{data}\PY{p}{)}
\end{Verbatim}
\end{tcolorbox}

    \begin{Verbatim}[commandchars=\\\{\}]
call now():
Today is: 2022-6-11
    \end{Verbatim}

    \hypertarget{ux903bux8f91ux8fd0ux7b97}{%
\subsection{逻辑运算}\label{ux903bux8f91ux8fd0ux7b97}}

    \hypertarget{ux6570ux5b57ux7f16ux7801}{%
\subsubsection{数字编码}\label{ux6570ux5b57ux7f16ux7801}}

在计算机中,我们可以用数字来编码\texttt{False}和\texttt{True}

\[
\text{False} \qquad \longleftrightarrow \qquad 0
\] \[
\text{True} \qquad \longleftrightarrow \qquad 1
\]

逻辑运算\texttt{not}、\texttt{and}和\texttt{or}也可以基于数字来实现

\[
\text{not}\ x \qquad \longleftrightarrow \qquad 1-x \quad \ \   \] \[
x \ \text{and}\ y \qquad \longleftrightarrow \qquad x y \qquad \quad \ \
\] \[
x \ \text{or}\ y \qquad \longleftrightarrow \qquad x + y - x y
\]

那么问题就来了,我们可不可以用函数来编码\texttt{True}和\texttt{False}呢?

    \hypertarget{true-false-ux7684ux5b9aux4e49}{%
\subsubsection{True, False 的定义}\label{true-false-ux7684ux5b9aux4e49}}

在传统函数思想中,True和False都是Bool变量,但在\(λ\)-演算中,所有的对象都是一个函数。

因此我们使用两个函数来表示True和False: \[
\text{TRUE}(x,y) \ = \ x, \quad \quad \text{FALSE}(x,y) \ = \ y
\] 为什么要这样定义呢?我们暂且不纠结这个问题,先来看下面这段代码

    \begin{tcolorbox}[breakable, size=fbox, boxrule=1pt, pad at break*=1mm,colback=cellbackground, colframe=cellborder]
\prompt{In}{incolor}{5}{\boxspacing}
\begin{Verbatim}[commandchars=\\\{\}]
\PY{k}{def} \PY{n+nf}{TRUE}\PY{p}{(}\PY{n}{x}\PY{p}{,}\PY{n}{y}\PY{p}{)}\PY{p}{:}
    \PY{k}{return} \PY{n}{x}

\PY{k}{def} \PY{n+nf}{FALSE}\PY{p}{(}\PY{n}{x}\PY{p}{,}\PY{n}{y}\PY{p}{)}\PY{p}{:}
    \PY{k}{return} \PY{n}{y}

\PY{n+nb}{print}\PY{p}{(}\PY{n}{TRUE}\PY{p}{(}\PY{n}{TRUE}\PY{p}{,}\PY{n}{FALSE}\PY{p}{)}\PY{o}{.}\PY{n+nv+vm}{\PYZus{}\PYZus{}name\PYZus{}\PYZus{}}\PY{p}{)}
\PY{n+nb}{print}\PY{p}{(}\PY{n}{TRUE}\PY{p}{(}\PY{n}{FALSE}\PY{p}{,}\PY{n}{TRUE}\PY{p}{)}\PY{o}{.}\PY{n+nv+vm}{\PYZus{}\PYZus{}name\PYZus{}\PYZus{}}\PY{p}{)}
\PY{n+nb}{print}\PY{p}{(}\PY{n}{FALSE}\PY{p}{(}\PY{n}{TRUE}\PY{p}{,}\PY{n}{FALSE}\PY{p}{)}\PY{o}{.}\PY{n+nv+vm}{\PYZus{}\PYZus{}name\PYZus{}\PYZus{}}\PY{p}{)}
\PY{n+nb}{print}\PY{p}{(}\PY{n}{FALSE}\PY{p}{(}\PY{n}{FALSE}\PY{p}{,}\PY{n}{TRUE}\PY{p}{)}\PY{o}{.}\PY{n+nv+vm}{\PYZus{}\PYZus{}name\PYZus{}\PYZus{}}\PY{p}{)}
\end{Verbatim}
\end{tcolorbox}

    \begin{Verbatim}[commandchars=\\\{\}]
TRUE
FALSE
FALSE
TRUE
    \end{Verbatim}

    我们看到了什么?\texttt{TRUE(TRUE,FALSE)}居然返回了函数\texttt{TRUE}本身!

是不是有点感觉了,我们写一个\texttt{show}函数来作为解码器

    \begin{tcolorbox}[breakable, size=fbox, boxrule=1pt, pad at break*=1mm,colback=cellbackground, colframe=cellborder]
\prompt{In}{incolor}{6}{\boxspacing}
\begin{Verbatim}[commandchars=\\\{\}]
\PY{k}{def} \PY{n+nf}{show}\PY{p}{(}\PY{n}{f}\PY{p}{)}\PY{p}{:}
    \PY{n+nb}{print}\PY{p}{(}\PY{n}{f}\PY{p}{(}\PY{n}{TRUE}\PY{p}{,}\PY{n}{FALSE}\PY{p}{)}\PY{o}{.}\PY{n+nv+vm}{\PYZus{}\PYZus{}name\PYZus{}\PYZus{}}\PY{p}{)}

\PY{n}{show}\PY{p}{(}\PY{n}{TRUE}\PY{p}{)}
\PY{n}{show}\PY{p}{(}\PY{n}{FALSE}\PY{p}{)}
\end{Verbatim}
\end{tcolorbox}

    \begin{Verbatim}[commandchars=\\\{\}]
TRUE
FALSE
    \end{Verbatim}

    \hypertarget{not-and-or}{%
\subsubsection{Not, And, Or}\label{not-and-or}}

根据 True和 False
的定义,我们将两个变量的位置进行交换,便可定义出NOT函数 \[
\text{NOT}(f)  \ = \ f(\text{FALSE},\text{TRUE}) 
\]

    \begin{tcolorbox}[breakable, size=fbox, boxrule=1pt, pad at break*=1mm,colback=cellbackground, colframe=cellborder]
\prompt{In}{incolor}{7}{\boxspacing}
\begin{Verbatim}[commandchars=\\\{\}]
\PY{k}{def} \PY{n+nf}{NOT}\PY{p}{(}\PY{n}{f}\PY{p}{)}\PY{p}{:}
    \PY{k}{return} \PY{n}{f}\PY{p}{(}\PY{n}{FALSE}\PY{p}{,}\PY{n}{TRUE}\PY{p}{)}

\PY{n}{show}\PY{p}{(}\PY{n}{NOT}\PY{p}{(}\PY{n}{TRUE}\PY{p}{)}\PY{p}{)}
\PY{n}{show}\PY{p}{(}\PY{n}{NOT}\PY{p}{(}\PY{n}{FALSE}\PY{p}{)}\PY{p}{)}
\end{Verbatim}
\end{tcolorbox}

    \begin{Verbatim}[commandchars=\\\{\}]
FALSE
TRUE
    \end{Verbatim}

    \texttt{NOT(TRUE)}返回了函数\texttt{FALSE},而\texttt{NOT(FALSE)}返回了函数\texttt{TRUE}!

那么你先不往下看,能自己想象出怎么定义\texttt{AND}函数吗?

    \begin{tcolorbox}[breakable, size=fbox, boxrule=1pt, pad at break*=1mm,colback=cellbackground, colframe=cellborder]
\prompt{In}{incolor}{8}{\boxspacing}
\begin{Verbatim}[commandchars=\\\{\}]
\PY{k}{def} \PY{n+nf}{AND}\PY{p}{(}\PY{n}{f}\PY{p}{,}\PY{n}{g}\PY{p}{)}\PY{p}{:}
    \PY{k}{pass}
\end{Verbatim}
\end{tcolorbox}

    \texttt{AND(f,g)==TRUE}
函数要求\texttt{f}和\texttt{g}都是TRUE,当\texttt{f}为FALSE时,会返回第二个参数,因为我们可以返回
\texttt{f(?,f)}。

如果\texttt{f}为TRUE,返回第一个参数中的\texttt{?},我们可以将\texttt{?}设为\texttt{g}本身,即

\[
\text{AND}(f,g) \ = \ f(\,g, f)
\]

    \begin{tcolorbox}[breakable, size=fbox, boxrule=1pt, pad at break*=1mm,colback=cellbackground, colframe=cellborder]
\prompt{In}{incolor}{9}{\boxspacing}
\begin{Verbatim}[commandchars=\\\{\}]
\PY{k}{def} \PY{n+nf}{AND}\PY{p}{(}\PY{n}{f}\PY{p}{,}\PY{n}{g}\PY{p}{)}\PY{p}{:} 
    \PY{k}{return} \PY{n}{f}\PY{p}{(}\PY{n}{g}\PY{p}{,}\PY{n}{f}\PY{p}{)}

\PY{n}{show}\PY{p}{(}\PY{n}{AND}\PY{p}{(}\PY{n}{TRUE}\PY{p}{,}\PY{n}{TRUE}\PY{p}{)}\PY{p}{)}
\PY{n}{show}\PY{p}{(}\PY{n}{AND}\PY{p}{(}\PY{n}{TRUE}\PY{p}{,}\PY{n}{FALSE}\PY{p}{)}\PY{p}{)}
\PY{n}{show}\PY{p}{(}\PY{n}{AND}\PY{p}{(}\PY{n}{FALSE}\PY{p}{,}\PY{n}{TRUE}\PY{p}{)}\PY{p}{)}
\PY{n}{show}\PY{p}{(}\PY{n}{AND}\PY{p}{(}\PY{n}{FALSE}\PY{p}{,}\PY{n}{FALSE}\PY{p}{)}\PY{p}{)}
\end{Verbatim}
\end{tcolorbox}

    \begin{Verbatim}[commandchars=\\\{\}]
TRUE
FALSE
FALSE
FALSE
    \end{Verbatim}

    OR函数也是类似的:

\[
\text{OR}(f,g) \ = \ f(f,g )
\]

    \begin{tcolorbox}[breakable, size=fbox, boxrule=1pt, pad at break*=1mm,colback=cellbackground, colframe=cellborder]
\prompt{In}{incolor}{10}{\boxspacing}
\begin{Verbatim}[commandchars=\\\{\}]
\PY{k}{def} \PY{n+nf}{OR}\PY{p}{(}\PY{n}{f}\PY{p}{,}\PY{n}{g}\PY{p}{)}\PY{p}{:}  
    \PY{k}{return} \PY{n}{f}\PY{p}{(}\PY{n}{f}\PY{p}{,}\PY{n}{g}\PY{p}{)}

\PY{k}{assert} \PY{n}{OR}\PY{p}{(}\PY{n}{TRUE}\PY{p}{,}\PY{n}{TRUE}\PY{p}{)} \PY{o+ow}{is} \PY{n}{TRUE}
\PY{k}{assert} \PY{n}{OR}\PY{p}{(}\PY{n}{TRUE}\PY{p}{,}\PY{n}{FALSE}\PY{p}{)} \PY{o+ow}{is} \PY{n}{TRUE}
\PY{k}{assert} \PY{n}{OR}\PY{p}{(}\PY{n}{FALSE}\PY{p}{,}\PY{n}{TRUE}\PY{p}{)} \PY{o+ow}{is} \PY{n}{TRUE}
\PY{k}{assert} \PY{n}{OR}\PY{p}{(}\PY{n}{FALSE}\PY{p}{,}\PY{n}{FALSE}\PY{p}{)} \PY{o+ow}{is} \PY{n}{FALSE}
\end{Verbatim}
\end{tcolorbox}

    \hypertarget{ux67efux91ccux5316currying}{%
\subsubsection{柯里化(Currying)}\label{ux67efux91ccux5316currying}}

柯里化就是将所有的函数变成只有一个参数,例如一个双参数函数\(f(x,y)\),我们将其变成一个二阶函数\(F(x)(y)\)

    \begin{tcolorbox}[breakable, size=fbox, boxrule=1pt, pad at break*=1mm,colback=cellbackground, colframe=cellborder]
\prompt{In}{incolor}{11}{\boxspacing}
\begin{Verbatim}[commandchars=\\\{\}]
\PY{k}{def} \PY{n+nf}{f}\PY{p}{(}\PY{n}{x}\PY{p}{,}\PY{n}{y}\PY{p}{)}\PY{p}{:}
    \PY{k}{return} \PY{n}{x}\PY{o}{+}\PY{n}{y}

\PY{k}{def} \PY{n+nf}{F}\PY{p}{(}\PY{n}{x}\PY{p}{)}\PY{p}{:}
    \PY{k}{def} \PY{n+nf}{Fx}\PY{p}{(}\PY{n}{y}\PY{p}{)}\PY{p}{:}
        \PY{k}{return} \PY{n}{x}\PY{o}{+}\PY{n}{y}
    \PY{k}{return} \PY{n}{Fx}

\PY{n+nb}{print}\PY{p}{(}\PY{n}{f}\PY{p}{(}\PY{l+s+s1}{\PYZsq{}}\PY{l+s+s1}{x}\PY{l+s+s1}{\PYZsq{}}\PY{p}{,}\PY{l+s+s1}{\PYZsq{}}\PY{l+s+s1}{y}\PY{l+s+s1}{\PYZsq{}}\PY{p}{)}\PY{p}{)}
\PY{n+nb}{print}\PY{p}{(}\PY{n}{F}\PY{p}{(}\PY{l+s+s1}{\PYZsq{}}\PY{l+s+s1}{x}\PY{l+s+s1}{\PYZsq{}}\PY{p}{)}\PY{p}{(}\PY{l+s+s1}{\PYZsq{}}\PY{l+s+s1}{y}\PY{l+s+s1}{\PYZsq{}}\PY{p}{)}\PY{p}{)}
\end{Verbatim}
\end{tcolorbox}

    \begin{Verbatim}[commandchars=\\\{\}]
xy
xy
    \end{Verbatim}

    我们可以将之前的逻辑函数都写成柯里化的形式

\[
N(f,x,y) \ = \ f(y,x) 
\]

\[
A(f,g,x,y) \ = \ f(\,g(x,y) , y)
\]

\[
O(f,g,x,y) \ = \ f(x,\,g(x,y) )
\]

    \begin{tcolorbox}[breakable, size=fbox, boxrule=1pt, pad at break*=1mm,colback=cellbackground, colframe=cellborder]
\prompt{In}{incolor}{12}{\boxspacing}
\begin{Verbatim}[commandchars=\\\{\}]
\PY{k}{def} \PY{n+nf}{T}\PY{p}{(}\PY{n}{x}\PY{p}{)}\PY{p}{:}
    \PY{k}{return} \PY{k}{lambda} \PY{n}{y}\PY{p}{:} \PY{n}{x}

\PY{k}{def} \PY{n+nf}{F}\PY{p}{(}\PY{n}{x}\PY{p}{)}\PY{p}{:}
    \PY{k}{return} \PY{k}{lambda} \PY{n}{y}\PY{p}{:} \PY{n}{y}

\PY{k}{def} \PY{n+nf}{N}\PY{p}{(}\PY{n}{f}\PY{p}{)}\PY{p}{:}
    \PY{k}{return} \PY{n}{f}\PY{p}{(}\PY{n}{F}\PY{p}{)}\PY{p}{(}\PY{n}{T}\PY{p}{)}

\PY{k}{def} \PY{n+nf}{A}\PY{p}{(}\PY{n}{f}\PY{p}{)}\PY{p}{:}
    \PY{k}{return} \PY{k}{lambda} \PY{n}{g}\PY{p}{:} \PY{n}{f}\PY{p}{(}\PY{n}{g}\PY{p}{)}\PY{p}{(}\PY{n}{f}\PY{p}{)}

\PY{k}{def} \PY{n+nf}{O}\PY{p}{(}\PY{n}{f}\PY{p}{)}\PY{p}{:}
    \PY{k}{return} \PY{k}{lambda} \PY{n}{g}\PY{p}{:} \PY{n}{f}\PY{p}{(}\PY{n}{f}\PY{p}{)}\PY{p}{(}\PY{n}{g}\PY{p}{)}

\PY{k}{assert} \PY{n}{T}\PY{p}{(}\PY{k+kc}{True}\PY{p}{)}\PY{p}{(}\PY{k+kc}{False}\PY{p}{)} \PY{o+ow}{is} \PY{k+kc}{True}
\PY{k}{assert} \PY{n}{F}\PY{p}{(}\PY{k+kc}{True}\PY{p}{)}\PY{p}{(}\PY{k+kc}{False}\PY{p}{)} \PY{o+ow}{is} \PY{k+kc}{False}
\PY{k}{assert} \PY{n}{N}\PY{p}{(}\PY{n}{T}\PY{p}{)} \PY{o+ow}{is} \PY{n}{F}
\PY{k}{assert} \PY{n}{N}\PY{p}{(}\PY{n}{F}\PY{p}{)} \PY{o+ow}{is} \PY{n}{T}
\PY{k}{assert} \PY{n}{A}\PY{p}{(}\PY{n}{T}\PY{p}{)}\PY{p}{(}\PY{n}{T}\PY{p}{)} \PY{o+ow}{is} \PY{n}{T}
\PY{k}{assert} \PY{n}{A}\PY{p}{(}\PY{n}{T}\PY{p}{)}\PY{p}{(}\PY{n}{F}\PY{p}{)} \PY{o+ow}{is} \PY{n}{F}
\PY{k}{assert} \PY{n}{O}\PY{p}{(}\PY{n}{T}\PY{p}{)}\PY{p}{(}\PY{n}{F}\PY{p}{)} \PY{o+ow}{is} \PY{n}{T}
\PY{k}{assert} \PY{n}{O}\PY{p}{(}\PY{n}{F}\PY{p}{)}\PY{p}{(}\PY{n}{F}\PY{p}{)} \PY{o+ow}{is} \PY{n}{F}
\end{Verbatim}
\end{tcolorbox}

    我们也可以写一个\texttt{curry}函数将\texttt{f(x,y)}转化成柯里化的版本\texttt{F(x)(y)}

    \begin{tcolorbox}[breakable, size=fbox, boxrule=1pt, pad at break*=1mm,colback=cellbackground, colframe=cellborder]
\prompt{In}{incolor}{13}{\boxspacing}
\begin{Verbatim}[commandchars=\\\{\}]
\PY{k}{def} \PY{n+nf}{curry}\PY{p}{(}\PY{n}{f}\PY{p}{)}\PY{p}{:}
    \PY{k}{return} \PY{k}{lambda} \PY{n}{x}\PY{p}{:} \PY{k}{lambda} \PY{n}{y}\PY{p}{:} \PY{n}{f}\PY{p}{(}\PY{n}{x}\PY{p}{,}\PY{n}{y}\PY{p}{)}

\PY{n}{F} \PY{o}{=} \PY{n}{curry}\PY{p}{(}\PY{n}{f}\PY{p}{)}
\PY{n+nb}{print}\PY{p}{(}\PY{n}{F}\PY{p}{(}\PY{l+s+s1}{\PYZsq{}}\PY{l+s+s1}{x}\PY{l+s+s1}{\PYZsq{}}\PY{p}{)}\PY{p}{(}\PY{l+s+s1}{\PYZsq{}}\PY{l+s+s1}{y}\PY{l+s+s1}{\PYZsq{}}\PY{p}{)}\PY{p}{)}
\end{Verbatim}
\end{tcolorbox}

    \begin{Verbatim}[commandchars=\\\{\}]
xy
    \end{Verbatim}

    在经过非平凡的思考(参考 https://zh.javascript.info/currying-partials
)之后,我们可以写出柯里化函数\texttt{curry}的通用形式:

    \begin{tcolorbox}[breakable, size=fbox, boxrule=1pt, pad at break*=1mm,colback=cellbackground, colframe=cellborder]
\prompt{In}{incolor}{14}{\boxspacing}
\begin{Verbatim}[commandchars=\\\{\}]
\PY{k}{def} \PY{n+nf}{curry}\PY{p}{(}\PY{n}{f}\PY{p}{)}\PY{p}{:}
    \PY{k}{def} \PY{n+nf}{curried}\PY{p}{(}\PY{o}{*}\PY{n}{args}\PY{p}{)}\PY{p}{:}
\PY{c+c1}{\PYZsh{}         print(\PYZsq{}\PYZhy{}\PYZhy{}\PYZhy{}\PYZsq{},*args,\PYZsq{}\PYZhy{}\PYZhy{}\PYZhy{}\PYZsq{})}
        \PY{k}{if} \PY{p}{(}\PY{n+nb}{len}\PY{p}{(}\PY{n}{args}\PY{p}{)} \PY{o}{==} \PY{n}{f}\PY{o}{.}\PY{n+nv+vm}{\PYZus{}\PYZus{}code\PYZus{}\PYZus{}}\PY{o}{.}\PY{n}{co\PYZus{}argcount}\PY{p}{)}\PY{p}{:}
            \PY{k}{return} \PY{n}{f}\PY{p}{(}\PY{o}{*}\PY{n}{args}\PY{p}{)}
        \PY{k}{else}\PY{p}{:}
            \PY{k}{return} \PY{k}{lambda} \PY{o}{*}\PY{n}{args2}\PY{p}{:} \PY{n}{curried}\PY{p}{(}\PY{o}{*}\PY{n}{args}\PY{p}{,}\PY{o}{*}\PY{n}{args2}\PY{p}{)}
    \PY{k}{return} \PY{n}{curried}
\end{Verbatim}
\end{tcolorbox}

    这里的\texttt{f.\_\_code\_\_.co\_argcount}返回\texttt{f}定义时的参数个数,例如\texttt{f3.\_\_code\_\_.co\_argcount=3}。

这段代码的思想就是当\texttt{f}的输入参数个数\texttt{len(args)}
与定义参数个数\texttt{f.\_\_code\_\_.co\_argcount}一致时,就直接输出\texttt{f(*args)}。

当参数不够时,就先把当前参数\texttt{*args}传进去,再递归调用外面一层的参数\texttt{*args2}。

    \begin{tcolorbox}[breakable, size=fbox, boxrule=1pt, pad at break*=1mm,colback=cellbackground, colframe=cellborder]
\prompt{In}{incolor}{15}{\boxspacing}
\begin{Verbatim}[commandchars=\\\{\}]
\PY{n}{f1} \PY{o}{=} \PY{k}{lambda} \PY{n}{x}\PY{p}{:} \PY{n}{x}
\PY{n}{F1} \PY{o}{=} \PY{n}{curry}\PY{p}{(}\PY{n}{f1}\PY{p}{)}
\PY{n+nb}{print}\PY{p}{(}\PY{n}{F1}\PY{p}{(}\PY{l+s+s1}{\PYZsq{}}\PY{l+s+s1}{x}\PY{l+s+s1}{\PYZsq{}}\PY{p}{)}\PY{p}{)}

\PY{n}{f2} \PY{o}{=} \PY{k}{lambda} \PY{n}{x}\PY{p}{,}\PY{n}{y}\PY{p}{:} \PY{n}{x}\PY{o}{+}\PY{n}{y}
\PY{n}{F2} \PY{o}{=} \PY{n}{curry}\PY{p}{(}\PY{n}{f2}\PY{p}{)}
\PY{n+nb}{print}\PY{p}{(}\PY{n}{F2}\PY{p}{(}\PY{l+s+s1}{\PYZsq{}}\PY{l+s+s1}{x}\PY{l+s+s1}{\PYZsq{}}\PY{p}{)}\PY{p}{(}\PY{l+s+s1}{\PYZsq{}}\PY{l+s+s1}{y}\PY{l+s+s1}{\PYZsq{}}\PY{p}{)}\PY{p}{)}

\PY{n}{f3} \PY{o}{=} \PY{k}{lambda} \PY{n}{x}\PY{p}{,}\PY{n}{y}\PY{p}{,}\PY{n}{z}\PY{p}{:} \PY{n}{x}\PY{o}{+}\PY{n}{y}\PY{o}{+}\PY{n}{z}
\PY{n}{F3} \PY{o}{=} \PY{n}{curry}\PY{p}{(}\PY{n}{f3}\PY{p}{)}
\PY{n+nb}{print}\PY{p}{(}\PY{n}{F3}\PY{p}{(}\PY{l+s+s1}{\PYZsq{}}\PY{l+s+s1}{x}\PY{l+s+s1}{\PYZsq{}}\PY{p}{)}\PY{p}{(}\PY{l+s+s1}{\PYZsq{}}\PY{l+s+s1}{y}\PY{l+s+s1}{\PYZsq{}}\PY{p}{)}\PY{p}{(}\PY{l+s+s1}{\PYZsq{}}\PY{l+s+s1}{z}\PY{l+s+s1}{\PYZsq{}}\PY{p}{)}\PY{p}{)}

\PY{n}{f4} \PY{o}{=} \PY{k}{lambda} \PY{n}{x}\PY{p}{,}\PY{n}{y}\PY{p}{,}\PY{n}{z}\PY{p}{,}\PY{n}{w}\PY{p}{:} \PY{n}{x}\PY{o}{+}\PY{n}{y}\PY{o}{+}\PY{n}{z}\PY{o}{+}\PY{n}{w}
\PY{n}{F4} \PY{o}{=} \PY{n}{curry}\PY{p}{(}\PY{n}{f4}\PY{p}{)}
\PY{n+nb}{print}\PY{p}{(}\PY{n}{F4}\PY{p}{(}\PY{l+m+mi}{1}\PY{p}{)}\PY{p}{(}\PY{l+m+mi}{2}\PY{p}{)}\PY{p}{(}\PY{l+m+mi}{3}\PY{p}{)}\PY{p}{(}\PY{l+m+mi}{4}\PY{p}{)}\PY{p}{)}
\PY{n+nb}{print}\PY{p}{(}\PY{n}{F4}\PY{p}{(}\PY{l+m+mi}{1}\PY{p}{,}\PY{l+m+mi}{2}\PY{p}{)}\PY{p}{(}\PY{l+m+mi}{3}\PY{p}{)}\PY{p}{(}\PY{l+m+mi}{4}\PY{p}{)}\PY{p}{)}
\PY{n+nb}{print}\PY{p}{(}\PY{n}{F4}\PY{p}{(}\PY{l+m+mi}{1}\PY{p}{,}\PY{l+m+mi}{2}\PY{p}{,}\PY{l+m+mi}{3}\PY{p}{,}\PY{l+m+mi}{4}\PY{p}{)}\PY{p}{)}
\end{Verbatim}
\end{tcolorbox}

    \begin{Verbatim}[commandchars=\\\{\}]
x
xy
xyz
10
10
10
    \end{Verbatim}

    如果你难以理解的话,可以取消掉\texttt{curry}函数中的注释,观察每次调用时传入的参数。

我们可以直接用\texttt{curry}函数将逻辑函数柯里化,当然,这和我们在上面直接定义的柯里化逻辑函数有着细微的区别,不过这并不重要。

    \begin{tcolorbox}[breakable, size=fbox, boxrule=1pt, pad at break*=1mm,colback=cellbackground, colframe=cellborder]
\prompt{In}{incolor}{16}{\boxspacing}
\begin{Verbatim}[commandchars=\\\{\}]
\PY{n}{T} \PY{o}{=} \PY{n}{curry}\PY{p}{(}\PY{n}{TRUE}\PY{p}{)}
\PY{n}{F} \PY{o}{=} \PY{n}{curry}\PY{p}{(}\PY{n}{FALSE}\PY{p}{)}
\PY{n}{N} \PY{o}{=} \PY{n}{curry}\PY{p}{(}\PY{n}{NOT}\PY{p}{)}
\PY{n}{A} \PY{o}{=} \PY{n}{curry}\PY{p}{(}\PY{n}{AND}\PY{p}{)}
\PY{n}{O} \PY{o}{=} \PY{n}{curry}\PY{p}{(}\PY{n}{OR}\PY{p}{)}

\PY{k}{assert} \PY{n}{T}\PY{p}{(}\PY{k+kc}{True}\PY{p}{)}\PY{p}{(}\PY{k+kc}{False}\PY{p}{)} \PY{o+ow}{is} \PY{k+kc}{True}
\PY{k}{assert} \PY{n}{F}\PY{p}{(}\PY{k+kc}{True}\PY{p}{)}\PY{p}{(}\PY{k+kc}{False}\PY{p}{)} \PY{o+ow}{is} \PY{k+kc}{False}
\PY{k}{assert} \PY{n}{N}\PY{p}{(}\PY{n}{T}\PY{p}{)} \PY{o+ow}{is} \PY{n}{FALSE}
\PY{k}{assert} \PY{n}{N}\PY{p}{(}\PY{n}{F}\PY{p}{)} \PY{o+ow}{is} \PY{n}{TRUE}
\PY{k}{assert} \PY{n}{A}\PY{p}{(}\PY{n}{T}\PY{p}{)}\PY{p}{(}\PY{n}{T}\PY{p}{)} \PY{o+ow}{is} \PY{n}{T}
\PY{k}{assert} \PY{n}{A}\PY{p}{(}\PY{n}{T}\PY{p}{)}\PY{p}{(}\PY{n}{F}\PY{p}{)} \PY{o+ow}{is} \PY{n}{F}
\PY{k}{assert} \PY{n}{O}\PY{p}{(}\PY{n}{T}\PY{p}{)}\PY{p}{(}\PY{n}{F}\PY{p}{)} \PY{o+ow}{is} \PY{n}{T}
\PY{k}{assert} \PY{n}{O}\PY{p}{(}\PY{n}{F}\PY{p}{)}\PY{p}{(}\PY{n}{F}\PY{p}{)} \PY{o+ow}{is} \PY{n}{F}

\PY{k}{def} \PY{n+nf}{show}\PY{p}{(}\PY{n}{f}\PY{p}{)}\PY{p}{:}
    \PY{n+nb}{print}\PY{p}{(}\PY{n}{f}\PY{p}{(}\PY{n}{TRUE}\PY{p}{,}\PY{n}{FALSE}\PY{p}{)}\PY{o}{.}\PY{n+nv+vm}{\PYZus{}\PYZus{}name\PYZus{}\PYZus{}}\PY{p}{)}
    
\PY{n}{show}\PY{p}{(}\PY{n}{T}\PY{p}{)}
\PY{n}{show}\PY{p}{(}\PY{n}{F}\PY{p}{)}
\PY{n}{show}\PY{p}{(}\PY{n}{N}\PY{p}{(}\PY{n}{T}\PY{p}{)}\PY{p}{)}
\PY{n}{show}\PY{p}{(}\PY{n}{A}\PY{p}{(}\PY{n}{T}\PY{p}{)}\PY{p}{(}\PY{n}{F}\PY{p}{)}\PY{p}{)}
\PY{n}{show}\PY{p}{(}\PY{n}{O}\PY{p}{(}\PY{n}{T}\PY{p}{)}\PY{p}{(}\PY{n}{F}\PY{p}{)}\PY{p}{)}
\end{Verbatim}
\end{tcolorbox}

    \begin{Verbatim}[commandchars=\\\{\}]
TRUE
FALSE
FALSE
FALSE
TRUE
    \end{Verbatim}

    \hypertarget{ux5bf9ux81eaux7136ux6570ux7f16ux7801}{%
\subsection{对自然数编码}\label{ux5bf9ux81eaux7136ux6570ux7f16ux7801}}

既然能用函数对逻辑运算编码,当然也可以对自然数编码,我们通过如下方式定义:

\begin{itemize}
\tightlist
\item
  每个自然数都是一个函数,它的输入和输出都是函数
\item
  函数0对于任何输入\(f(\cdot)\),输出都是一个恒等函数\(x\to x\)
\item
  函数1对于输入\(f(\cdot)\),返回\(f\)本身
\item
  函数2对于输入\(f(\cdot)\),返回\(f\)的二阶函数\(f(f(\cdot))\)
\end{itemize}

    \begin{tcolorbox}[breakable, size=fbox, boxrule=1pt, pad at break*=1mm,colback=cellbackground, colframe=cellborder]
\prompt{In}{incolor}{17}{\boxspacing}
\begin{Verbatim}[commandchars=\\\{\}]
\PY{k}{def}  \PY{n+nf}{ZERO}\PY{p}{(}\PY{n}{f}\PY{p}{)}\PY{p}{:} \PY{k}{return} \PY{k}{lambda} \PY{n}{x}\PY{p}{:} \PY{n}{x}
\PY{k}{def}   \PY{n+nf}{ONE}\PY{p}{(}\PY{n}{f}\PY{p}{)}\PY{p}{:} \PY{k}{return} \PY{k}{lambda} \PY{n}{x}\PY{p}{:} \PY{n}{f}\PY{p}{(}\PY{n}{x}\PY{p}{)}
\PY{k}{def}   \PY{n+nf}{TWO}\PY{p}{(}\PY{n}{f}\PY{p}{)}\PY{p}{:} \PY{k}{return} \PY{k}{lambda} \PY{n}{x}\PY{p}{:} \PY{n}{f}\PY{p}{(}\PY{n}{f}\PY{p}{(}\PY{n}{x}\PY{p}{)}\PY{p}{)}
\PY{k}{def} \PY{n+nf}{THREE}\PY{p}{(}\PY{n}{f}\PY{p}{)}\PY{p}{:} \PY{k}{return} \PY{k}{lambda} \PY{n}{x}\PY{p}{:} \PY{n}{f}\PY{p}{(}\PY{n}{f}\PY{p}{(}\PY{n}{f}\PY{p}{(}\PY{n}{x}\PY{p}{)}\PY{p}{)}\PY{p}{)}
\PY{k}{def}  \PY{n+nf}{FOUR}\PY{p}{(}\PY{n}{f}\PY{p}{)}\PY{p}{:} \PY{k}{return} \PY{k}{lambda} \PY{n}{x}\PY{p}{:} \PY{n}{f}\PY{p}{(}\PY{n}{f}\PY{p}{(}\PY{n}{f}\PY{p}{(}\PY{n}{f}\PY{p}{(}\PY{n}{x}\PY{p}{)}\PY{p}{)}\PY{p}{)}\PY{p}{)}
\PY{k}{def}  \PY{n+nf}{FIVE}\PY{p}{(}\PY{n}{f}\PY{p}{)}\PY{p}{:} \PY{k}{return} \PY{k}{lambda} \PY{n}{x}\PY{p}{:} \PY{n}{f}\PY{p}{(}\PY{n}{f}\PY{p}{(}\PY{n}{f}\PY{p}{(}\PY{n}{f}\PY{p}{(}\PY{n}{f}\PY{p}{(}\PY{n}{x}\PY{p}{)}\PY{p}{)}\PY{p}{)}\PY{p}{)}\PY{p}{)}
\PY{k}{def}   \PY{n+nf}{SIX}\PY{p}{(}\PY{n}{f}\PY{p}{)}\PY{p}{:} \PY{k}{return} \PY{k}{lambda} \PY{n}{x}\PY{p}{:} \PY{n}{f}\PY{p}{(}\PY{n}{f}\PY{p}{(}\PY{n}{f}\PY{p}{(}\PY{n}{f}\PY{p}{(}\PY{n}{f}\PY{p}{(}\PY{n}{f}\PY{p}{(}\PY{n}{x}\PY{p}{)}\PY{p}{)}\PY{p}{)}\PY{p}{)}\PY{p}{)}\PY{p}{)}
\end{Verbatim}
\end{tcolorbox}

    \begin{tcolorbox}[breakable, size=fbox, boxrule=1pt, pad at break*=1mm,colback=cellbackground, colframe=cellborder]
\prompt{In}{incolor}{18}{\boxspacing}
\begin{Verbatim}[commandchars=\\\{\}]
\PY{n}{hi} \PY{o}{=} \PY{k}{lambda} \PY{n}{x}\PY{p}{:} \PY{n}{x} \PY{o}{+} \PY{l+s+s1}{\PYZsq{}}\PY{l+s+s1}{ World}\PY{l+s+s1}{\PYZsq{}}
\PY{n}{x} \PY{o}{=} \PY{l+s+s1}{\PYZsq{}}\PY{l+s+s1}{Hello,}\PY{l+s+s1}{\PYZsq{}}

\PY{n+nb}{print}\PY{p}{(}\PY{n}{ZERO}\PY{p}{(}\PY{n}{hi}\PY{p}{)}\PY{p}{(}\PY{n}{x}\PY{p}{)}\PY{p}{)}
\PY{n+nb}{print}\PY{p}{(}\PY{n}{ONE}\PY{p}{(}\PY{n}{hi}\PY{p}{)}\PY{p}{(}\PY{n}{x}\PY{p}{)}\PY{p}{)}
\PY{n+nb}{print}\PY{p}{(}\PY{n}{TWO}\PY{p}{(}\PY{n}{hi}\PY{p}{)}\PY{p}{(}\PY{n}{x}\PY{p}{)}\PY{p}{)}
\PY{n+nb}{print}\PY{p}{(}\PY{n}{THREE}\PY{p}{(}\PY{n}{hi}\PY{p}{)}\PY{p}{(}\PY{n}{x}\PY{p}{)}\PY{p}{)}
\PY{n+nb}{print}\PY{p}{(}\PY{n}{FOUR}\PY{p}{(}\PY{n}{hi}\PY{p}{)}\PY{p}{(}\PY{n}{x}\PY{p}{)}\PY{p}{)}
\PY{n+nb}{print}\PY{p}{(}\PY{n}{FIVE}\PY{p}{(}\PY{n}{hi}\PY{p}{)}\PY{p}{(}\PY{n}{x}\PY{p}{)}\PY{p}{)}
\PY{n+nb}{print}\PY{p}{(}\PY{n}{SIX}\PY{p}{(}\PY{n}{hi}\PY{p}{)}\PY{p}{(}\PY{n}{x}\PY{p}{)}\PY{p}{)}
\end{Verbatim}
\end{tcolorbox}

    \begin{Verbatim}[commandchars=\\\{\}]
Hello,
Hello, World
Hello, World World
Hello, World World World
Hello, World World World World
Hello, World World World World World
Hello, World World World World World World
    \end{Verbatim}

    如果我们定义零元为0,每一个后继就是前驱元素+1,那么就能得到常规形式的自然数:

    \begin{tcolorbox}[breakable, size=fbox, boxrule=1pt, pad at break*=1mm,colback=cellbackground, colframe=cellborder]
\prompt{In}{incolor}{19}{\boxspacing}
\begin{Verbatim}[commandchars=\\\{\}]
\PY{k}{def} \PY{n+nf}{Num}\PY{p}{(}\PY{n}{x}\PY{p}{)}\PY{p}{:}
    \PY{k}{return} \PY{n}{x} \PY{o}{+} \PY{l+m+mi}{1}
\PY{n}{x} \PY{o}{=} \PY{l+m+mi}{0}
\PY{n+nb}{print}\PY{p}{(}\PY{n}{ZERO}\PY{p}{(}\PY{n}{Num}\PY{p}{)}\PY{p}{(}\PY{n}{x}\PY{p}{)}\PY{p}{)}
\PY{n+nb}{print}\PY{p}{(}\PY{n}{ONE}\PY{p}{(}\PY{n}{Num}\PY{p}{)}\PY{p}{(}\PY{n}{x}\PY{p}{)}\PY{p}{)}
\PY{n+nb}{print}\PY{p}{(}\PY{n}{TWO}\PY{p}{(}\PY{n}{Num}\PY{p}{)}\PY{p}{(}\PY{n}{x}\PY{p}{)}\PY{p}{)}
\PY{n+nb}{print}\PY{p}{(}\PY{n}{THREE}\PY{p}{(}\PY{n}{Num}\PY{p}{)}\PY{p}{(}\PY{n}{x}\PY{p}{)}\PY{p}{)}
\end{Verbatim}
\end{tcolorbox}

    \begin{Verbatim}[commandchars=\\\{\}]
0
1
2
3
    \end{Verbatim}

    我们借此将函数版的自然数转化成常规形式的自然数

    \begin{tcolorbox}[breakable, size=fbox, boxrule=1pt, pad at break*=1mm,colback=cellbackground, colframe=cellborder]
\prompt{In}{incolor}{20}{\boxspacing}
\begin{Verbatim}[commandchars=\\\{\}]
\PY{k}{def} \PY{n+nf}{show}\PY{p}{(}\PY{n}{x}\PY{p}{)}\PY{p}{:}
    \PY{n+nb}{print}\PY{p}{(}\PY{n}{x}\PY{p}{(}\PY{k}{lambda} \PY{n}{x}\PY{p}{:} \PY{n}{x}\PY{o}{+}\PY{l+m+mi}{1}\PY{p}{)}\PY{p}{(}\PY{l+m+mi}{0}\PY{p}{)}\PY{p}{)}

\PY{n}{show}\PY{p}{(}\PY{n}{ZERO}\PY{p}{)}
\PY{n}{show}\PY{p}{(}\PY{n}{ONE}\PY{p}{)}
\PY{n}{show}\PY{p}{(}\PY{n}{TWO}\PY{p}{)}
\PY{n}{show}\PY{p}{(}\PY{n}{THREE}\PY{p}{)}
\end{Verbatim}
\end{tcolorbox}

    \begin{Verbatim}[commandchars=\\\{\}]
0
1
2
3
    \end{Verbatim}

    \hypertarget{next-number}{%
\subsubsection{Next Number}\label{next-number}}

显然,我们不可能手动定义所有的自然数函数,因此我们需要定义一个''后继函数'':对于任意一个自然数\texttt{A},我们定义它的后继为
\texttt{NEXT(A)}。

我们可以先写出一个非柯里化版本:

    \begin{tcolorbox}[breakable, size=fbox, boxrule=1pt, pad at break*=1mm,colback=cellbackground, colframe=cellborder]
\prompt{In}{incolor}{21}{\boxspacing}
\begin{Verbatim}[commandchars=\\\{\}]
\PY{k}{def} \PY{n+nf}{NEXT}\PY{p}{(}\PY{n}{A}\PY{p}{,}\PY{n}{f}\PY{p}{,}\PY{n}{x}\PY{p}{)}\PY{p}{:}
    \PY{k}{return} \PY{n}{f}\PY{p}{(}\PY{n}{A}\PY{p}{(}\PY{n}{f}\PY{p}{)}\PY{p}{(}\PY{n}{x}\PY{p}{)}\PY{p}{)}

\PY{n}{fish} \PY{o}{=} \PY{k}{lambda} \PY{n}{x}\PY{p}{:} \PY{n}{x} \PY{o}{+} \PY{l+s+s1}{\PYZsq{}}\PY{l+s+s1}{ fish}\PY{l+s+s1}{\PYZsq{}}
\PY{n}{x} \PY{o}{=} \PY{l+s+s1}{\PYZsq{}}\PY{l+s+s1}{I want:}\PY{l+s+s1}{\PYZsq{}}
\PY{n+nb}{print}\PY{p}{(}\PY{n}{NEXT}\PY{p}{(}\PY{n}{THREE}\PY{p}{,}\PY{n}{fish}\PY{p}{,}\PY{n}{x}\PY{p}{)}\PY{p}{)}
\end{Verbatim}
\end{tcolorbox}

    \begin{Verbatim}[commandchars=\\\{\}]
I want: fish fish fish fish
    \end{Verbatim}

    如果要柯里化,那么我们可以使用前面定义的\texttt{curry}函数:

    \begin{tcolorbox}[breakable, size=fbox, boxrule=1pt, pad at break*=1mm,colback=cellbackground, colframe=cellborder]
\prompt{In}{incolor}{22}{\boxspacing}
\begin{Verbatim}[commandchars=\\\{\}]
\PY{n}{NEXT\PYZus{}c} \PY{o}{=} \PY{n}{curry}\PY{p}{(}\PY{n}{NEXT}\PY{p}{)}
\PY{n+nb}{print}\PY{p}{(}\PY{n}{NEXT\PYZus{}c}\PY{p}{(}\PY{n}{THREE}\PY{p}{)}\PY{p}{(}\PY{n}{fish}\PY{p}{)}\PY{p}{(}\PY{n}{x}\PY{p}{)}\PY{p}{)}
\PY{n}{show}\PY{p}{(}\PY{n}{NEXT\PYZus{}c}\PY{p}{(}\PY{n}{THREE}\PY{p}{)}\PY{p}{)}
\end{Verbatim}
\end{tcolorbox}

    \begin{Verbatim}[commandchars=\\\{\}]
I want: fish fish fish fish
4
    \end{Verbatim}

    手写柯里化版本也很简单:

    \begin{tcolorbox}[breakable, size=fbox, boxrule=1pt, pad at break*=1mm,colback=cellbackground, colframe=cellborder]
\prompt{In}{incolor}{23}{\boxspacing}
\begin{Verbatim}[commandchars=\\\{\}]
\PY{k}{def} \PY{n+nf}{NEXT}\PY{p}{(}\PY{n}{A}\PY{p}{)}\PY{p}{:}
    \PY{k}{return} \PY{k}{lambda} \PY{n}{f}\PY{p}{:} \PY{k}{lambda} \PY{n}{x}\PY{p}{:} \PY{n}{f}\PY{p}{(}\PY{n}{A}\PY{p}{(}\PY{n}{f}\PY{p}{)}\PY{p}{(}\PY{n}{x}\PY{p}{)}\PY{p}{)}

\PY{n}{show}\PY{p}{(}\PY{n}{NEXT}\PY{p}{(}\PY{n}{THREE}\PY{p}{)}\PY{p}{)}
\end{Verbatim}
\end{tcolorbox}

    \begin{Verbatim}[commandchars=\\\{\}]
4
    \end{Verbatim}

    \hypertarget{ux52a0ux6cd5}{%
\subsubsection{加法}\label{ux52a0ux6cd5}}

我们要计算\texttt{A+B},以A为起点,计算B次NEXT即可

    \begin{tcolorbox}[breakable, size=fbox, boxrule=1pt, pad at break*=1mm,colback=cellbackground, colframe=cellborder]
\prompt{In}{incolor}{24}{\boxspacing}
\begin{Verbatim}[commandchars=\\\{\}]
\PY{k}{def} \PY{n+nf}{ADD}\PY{p}{(}\PY{n}{A}\PY{p}{,}\PY{n}{B}\PY{p}{)}\PY{p}{:}
    \PY{k}{return} \PY{n}{B}\PY{p}{(}\PY{n}{NEXT}\PY{p}{)}\PY{p}{(}\PY{n}{A}\PY{p}{)}

\PY{n}{show}\PY{p}{(}\PY{n}{ADD}\PY{p}{(}\PY{n}{TWO}\PY{p}{,}\PY{n}{THREE}\PY{p}{)}\PY{p}{)}
\end{Verbatim}
\end{tcolorbox}

    \begin{Verbatim}[commandchars=\\\{\}]
5
    \end{Verbatim}

    显然我们的加法是满足交换律的

    \begin{tcolorbox}[breakable, size=fbox, boxrule=1pt, pad at break*=1mm,colback=cellbackground, colframe=cellborder]
\prompt{In}{incolor}{25}{\boxspacing}
\begin{Verbatim}[commandchars=\\\{\}]
\PY{n}{show}\PY{p}{(}\PY{n}{ADD}\PY{p}{(}\PY{n}{ZERO}\PY{p}{,}\PY{n}{SIX}\PY{p}{)}\PY{p}{)}
\PY{n}{show}\PY{p}{(}\PY{n}{ADD}\PY{p}{(}\PY{n}{ONE}\PY{p}{,}\PY{n}{FIVE}\PY{p}{)}\PY{p}{)}
\PY{n}{show}\PY{p}{(}\PY{n}{ADD}\PY{p}{(}\PY{n}{TWO}\PY{p}{,}\PY{n}{FOUR}\PY{p}{)}\PY{p}{)}
\PY{n}{show}\PY{p}{(}\PY{n}{ADD}\PY{p}{(}\PY{n}{THREE}\PY{p}{,}\PY{n}{THREE}\PY{p}{)}\PY{p}{)}
\PY{n}{show}\PY{p}{(}\PY{n}{ADD}\PY{p}{(}\PY{n}{FOUR}\PY{p}{,}\PY{n}{TWO}\PY{p}{)}\PY{p}{)}
\PY{n}{show}\PY{p}{(}\PY{n}{ADD}\PY{p}{(}\PY{n}{FIVE}\PY{p}{,}\PY{n}{ONE}\PY{p}{)}\PY{p}{)}
\end{Verbatim}
\end{tcolorbox}

    \begin{Verbatim}[commandchars=\\\{\}]
6
6
6
6
6
6
    \end{Verbatim}

    \hypertarget{ux4e58ux6cd5}{%
\subsubsection{乘法}\label{ux4e58ux6cd5}}

要计算\texttt{A*B},那么只需累加A次\texttt{B(f)}即可

    \begin{tcolorbox}[breakable, size=fbox, boxrule=1pt, pad at break*=1mm,colback=cellbackground, colframe=cellborder]
\prompt{In}{incolor}{26}{\boxspacing}
\begin{Verbatim}[commandchars=\\\{\}]
\PY{k}{def} \PY{n+nf}{MULTIPLY}\PY{p}{(}\PY{n}{A}\PY{p}{,}\PY{n}{B}\PY{p}{)}\PY{p}{:}
    \PY{k}{return} \PY{k}{lambda} \PY{n}{f}\PY{p}{:} \PY{n}{A}\PY{p}{(}\PY{n}{B}\PY{p}{(}\PY{n}{f}\PY{p}{)}\PY{p}{)}

\PY{n}{show}\PY{p}{(}\PY{n}{MULTIPLY}\PY{p}{(}\PY{n}{FOUR}\PY{p}{,}\PY{n}{THREE}\PY{p}{)}\PY{p}{)}
\PY{n}{show}\PY{p}{(}\PY{n}{MULTIPLY}\PY{p}{(}\PY{n}{THREE}\PY{p}{,}\PY{n}{FOUR}\PY{p}{)}\PY{p}{)}
\end{Verbatim}
\end{tcolorbox}

    \begin{Verbatim}[commandchars=\\\{\}]
12
12
    \end{Verbatim}

    另一种思路是执行A次\texttt{ADD(0,B)}

    \begin{tcolorbox}[breakable, size=fbox, boxrule=1pt, pad at break*=1mm,colback=cellbackground, colframe=cellborder]
\prompt{In}{incolor}{27}{\boxspacing}
\begin{Verbatim}[commandchars=\\\{\}]
\PY{k}{def} \PY{n+nf}{MULTIPLY}\PY{p}{(}\PY{n}{A}\PY{p}{,}\PY{n}{B}\PY{p}{)}\PY{p}{:}
    \PY{k}{def} \PY{n+nf}{ADD\PYZus{}B}\PY{p}{(}\PY{n}{C}\PY{p}{)}\PY{p}{:}
        \PY{k}{return} \PY{n}{ADD}\PY{p}{(}\PY{n}{C}\PY{p}{,}\PY{n}{B}\PY{p}{)}
    \PY{k}{return} \PY{n}{A}\PY{p}{(}\PY{n}{ADD\PYZus{}B}\PY{p}{)}\PY{p}{(}\PY{n}{ZERO}\PY{p}{)}

\PY{n}{show}\PY{p}{(}\PY{n}{MULTIPLY}\PY{p}{(}\PY{n}{FOUR}\PY{p}{,}\PY{n}{THREE}\PY{p}{)}\PY{p}{)}
\PY{n}{show}\PY{p}{(}\PY{n}{MULTIPLY}\PY{p}{(}\PY{n}{THREE}\PY{p}{,}\PY{n}{FOUR}\PY{p}{)}\PY{p}{)}
\end{Verbatim}
\end{tcolorbox}

    \begin{Verbatim}[commandchars=\\\{\}]
12
12
    \end{Verbatim}

    \hypertarget{ux6307ux6570}{%
\subsubsection{指数}\label{ux6307ux6570}}

\texttt{A**B} 就是将A重复执行B次

    \begin{tcolorbox}[breakable, size=fbox, boxrule=1pt, pad at break*=1mm,colback=cellbackground, colframe=cellborder]
\prompt{In}{incolor}{28}{\boxspacing}
\begin{Verbatim}[commandchars=\\\{\}]
\PY{k}{def} \PY{n+nf}{POWER}\PY{p}{(}\PY{n}{A}\PY{p}{,}\PY{n}{B}\PY{p}{)}\PY{p}{:}
    \PY{k}{return} \PY{n}{B}\PY{p}{(}\PY{n}{A}\PY{p}{)}

\PY{n}{show}\PY{p}{(}\PY{n}{POWER}\PY{p}{(}\PY{n}{FOUR}\PY{p}{,}\PY{n}{THREE}\PY{p}{)}\PY{p}{)}
\PY{n}{show}\PY{p}{(}\PY{n}{POWER}\PY{p}{(}\PY{n}{THREE}\PY{p}{,}\PY{n}{FOUR}\PY{p}{)}\PY{p}{)}
\end{Verbatim}
\end{tcolorbox}

    \begin{Verbatim}[commandchars=\\\{\}]
64
81
    \end{Verbatim}

    \hypertarget{ux51cfux6cd5}{%
\subsubsection{减法}\label{ux51cfux6cd5}}

关于减法,这并非一件简单的事情,我们首先要考虑的问题是怎么计算
\texttt{N-ONE},也就是找到A的前驱,除非我们知道\texttt{f}的逆函数,并且逆是唯一的。

但对于一般的情况,我们的做法是定义一个函数 \[
\Phi  \ : \ (a,b) \ \to \ (b,b+1)
\] 对于\texttt{(ZERO,ZERO)}作用N次就得到了 \[
\Phi^{N}  \ : \ (0,0) \ \to \ (N-1,N)
\] 我们取第一个元素就得到了\texttt{N-ONE}

    \begin{tcolorbox}[breakable, size=fbox, boxrule=1pt, pad at break*=1mm,colback=cellbackground, colframe=cellborder]
\prompt{In}{incolor}{29}{\boxspacing}
\begin{Verbatim}[commandchars=\\\{\}]
\PY{k}{def} \PY{n+nf}{PAIR}\PY{p}{(}\PY{n}{A}\PY{p}{,}\PY{n}{B}\PY{p}{)}\PY{p}{:}
    \PY{k}{return} \PY{k}{lambda} \PY{n}{P}\PY{p}{:} \PY{n}{P}\PY{p}{(}\PY{n}{A}\PY{p}{,}\PY{n}{B}\PY{p}{)}

\PY{k}{def} \PY{n+nf}{PHI}\PY{p}{(}\PY{n}{P}\PY{p}{)}\PY{p}{:}
    \PY{n}{B} \PY{o}{=} \PY{n}{P}\PY{p}{(}\PY{n}{FALSE}\PY{p}{)}
    \PY{k}{return} \PY{n}{PAIR}\PY{p}{(}\PY{n}{B}\PY{p}{,}\PY{n}{NEXT}\PY{p}{(}\PY{n}{B}\PY{p}{)}\PY{p}{)}
    
\PY{k}{def} \PY{n+nf}{PRIOR}\PY{p}{(}\PY{n}{N}\PY{p}{)}\PY{p}{:}
    \PY{n}{PHI\PYZus{}N} \PY{o}{=} \PY{n}{POWER}\PY{p}{(}\PY{n}{PHI}\PY{p}{,}\PY{n}{N}\PY{p}{)}
    \PY{k}{return} \PY{n}{PHI\PYZus{}N}\PY{p}{(}\PY{n}{PAIR}\PY{p}{(}\PY{n}{ZERO}\PY{p}{,}\PY{n}{ZERO}\PY{p}{)}\PY{p}{)}\PY{p}{(}\PY{n}{TRUE}\PY{p}{)}

\PY{n}{show}\PY{p}{(}\PY{n}{PRIOR}\PY{p}{(}\PY{n}{FOUR}\PY{p}{)}\PY{p}{)}
\end{Verbatim}
\end{tcolorbox}

    \begin{Verbatim}[commandchars=\\\{\}]
3
    \end{Verbatim}

    在定义了前驱之后,就可以类似于加法一样定义减法了

    \begin{tcolorbox}[breakable, size=fbox, boxrule=1pt, pad at break*=1mm,colback=cellbackground, colframe=cellborder]
\prompt{In}{incolor}{30}{\boxspacing}
\begin{Verbatim}[commandchars=\\\{\}]
\PY{k}{def} \PY{n+nf}{SUBTRACT}\PY{p}{(}\PY{n}{A}\PY{p}{,}\PY{n}{B}\PY{p}{)}\PY{p}{:}
    \PY{k}{return} \PY{n}{B}\PY{p}{(}\PY{n}{PRIOR}\PY{p}{)}\PY{p}{(}\PY{n}{A}\PY{p}{)}

\PY{n}{show}\PY{p}{(}\PY{n}{SUBTRACT}\PY{p}{(}\PY{n}{SIX}\PY{p}{,}\PY{n}{TWO}\PY{p}{)}\PY{p}{)}
\PY{n}{show}\PY{p}{(}\PY{n}{SUBTRACT}\PY{p}{(}\PY{n}{POWER}\PY{p}{(}\PY{n}{TWO}\PY{p}{,}\PY{n}{FOUR}\PY{p}{)}\PY{p}{,}\PY{n}{THREE}\PY{p}{)}\PY{p}{)}
\end{Verbatim}
\end{tcolorbox}

    \begin{Verbatim}[commandchars=\\\{\}]
4
13
    \end{Verbatim}

    当然,这也存在着一些问题,因为我们是从零元开始的,并没有定义零元之前的''负数'',因此用小数减大数得到的依然为零元

    \begin{tcolorbox}[breakable, size=fbox, boxrule=1pt, pad at break*=1mm,colback=cellbackground, colframe=cellborder]
\prompt{In}{incolor}{31}{\boxspacing}
\begin{Verbatim}[commandchars=\\\{\}]
\PY{n}{show}\PY{p}{(}\PY{n}{SUBTRACT}\PY{p}{(}\PY{n}{TWO}\PY{p}{,}\PY{n}{SIX}\PY{p}{)}\PY{p}{)}
\end{Verbatim}
\end{tcolorbox}

    \begin{Verbatim}[commandchars=\\\{\}]
0
    \end{Verbatim}

    \hypertarget{ux5224ux65adux96f6ux5143}{%
\subsubsection{判断零元}\label{ux5224ux65adux96f6ux5143}}

由零函数的定义可知,对于任何\texttt{f},都有\texttt{ZERO(f)(x)==x}

    \begin{tcolorbox}[breakable, size=fbox, boxrule=1pt, pad at break*=1mm,colback=cellbackground, colframe=cellborder]
\prompt{In}{incolor}{32}{\boxspacing}
\begin{Verbatim}[commandchars=\\\{\}]
\PY{n}{f} \PY{o}{=} \PY{k}{lambda} \PY{n}{x}\PY{p}{:} \PY{n}{x} \PY{o}{+} \PY{l+s+s1}{\PYZsq{}}\PY{l+s+s1}{ world}\PY{l+s+s1}{\PYZsq{}}
\PY{n}{x} \PY{o}{=} \PY{l+s+s1}{\PYZsq{}}\PY{l+s+s1}{Hello,}\PY{l+s+s1}{\PYZsq{}}
\PY{n+nb}{print}\PY{p}{(}\PY{n}{ZERO}\PY{p}{(}\PY{n}{f}\PY{p}{)}\PY{p}{(}\PY{n}{x}\PY{p}{)}\PY{p}{)}
\PY{n+nb}{print}\PY{p}{(}\PY{n}{ONE}\PY{p}{(}\PY{n}{f}\PY{p}{)}\PY{p}{(}\PY{n}{x}\PY{p}{)}\PY{p}{)}
\PY{n+nb}{print}\PY{p}{(}\PY{n}{TWO}\PY{p}{(}\PY{n}{f}\PY{p}{)}\PY{p}{(}\PY{n}{x}\PY{p}{)}\PY{p}{)}
\PY{n+nb}{print}\PY{p}{(}\PY{n}{THREE}\PY{p}{(}\PY{n}{f}\PY{p}{)}\PY{p}{(}\PY{n}{x}\PY{p}{)}\PY{p}{)}
\end{Verbatim}
\end{tcolorbox}

    \begin{Verbatim}[commandchars=\\\{\}]
Hello,
Hello, world
Hello, world world
Hello, world world world
    \end{Verbatim}

    因此我们可以让\texttt{f}总是返回FALSE,让x为TRUE,那么只有零函数会返回TRUE,其他自然数都返回FALSE

    \begin{tcolorbox}[breakable, size=fbox, boxrule=1pt, pad at break*=1mm,colback=cellbackground, colframe=cellborder]
\prompt{In}{incolor}{33}{\boxspacing}
\begin{Verbatim}[commandchars=\\\{\}]
\PY{n}{f} \PY{o}{=} \PY{k}{lambda} \PY{n}{x}\PY{p}{:} \PY{n}{FALSE}
\PY{n}{x} \PY{o}{=} \PY{n}{TRUE}
\PY{n+nb}{print}\PY{p}{(}\PY{n}{ZERO}\PY{p}{(}\PY{n}{f}\PY{p}{)}\PY{p}{(}\PY{n}{x}\PY{p}{)}\PY{p}{)}
\PY{n+nb}{print}\PY{p}{(}\PY{n}{ONE}\PY{p}{(}\PY{n}{f}\PY{p}{)}\PY{p}{(}\PY{n}{x}\PY{p}{)}\PY{p}{)}
\PY{n+nb}{print}\PY{p}{(}\PY{n}{TWO}\PY{p}{(}\PY{n}{f}\PY{p}{)}\PY{p}{(}\PY{n}{x}\PY{p}{)}\PY{p}{)}
\PY{n+nb}{print}\PY{p}{(}\PY{n}{THREE}\PY{p}{(}\PY{n}{f}\PY{p}{)}\PY{p}{(}\PY{n}{x}\PY{p}{)}\PY{p}{)}
\end{Verbatim}
\end{tcolorbox}

    \begin{Verbatim}[commandchars=\\\{\}]
<function TRUE at 0x1064e5550>
<function FALSE at 0x1064e5310>
<function FALSE at 0x1064e5310>
<function FALSE at 0x1064e5310>
    \end{Verbatim}

    据此可写出判断零元的函数

    \begin{tcolorbox}[breakable, size=fbox, boxrule=1pt, pad at break*=1mm,colback=cellbackground, colframe=cellborder]
\prompt{In}{incolor}{34}{\boxspacing}
\begin{Verbatim}[commandchars=\\\{\}]
\PY{k}{def} \PY{n+nf}{ISZERO}\PY{p}{(}\PY{n}{N}\PY{p}{)}\PY{p}{:}
    \PY{n}{f} \PY{o}{=} \PY{k}{lambda} \PY{n}{x}\PY{p}{:} \PY{n}{FALSE}
    \PY{k}{return} \PY{n}{N}\PY{p}{(}\PY{n}{f}\PY{p}{)}\PY{p}{(}\PY{n}{TRUE}\PY{p}{)}

\PY{n+nb}{print}\PY{p}{(}\PY{n}{ISZERO}\PY{p}{(}\PY{n}{ZERO}\PY{p}{)}\PY{o}{.}\PY{n+nv+vm}{\PYZus{}\PYZus{}name\PYZus{}\PYZus{}}\PY{p}{)}
\PY{n+nb}{print}\PY{p}{(}\PY{n}{ISZERO}\PY{p}{(}\PY{n}{ONE}\PY{p}{)}\PY{o}{.}\PY{n+nv+vm}{\PYZus{}\PYZus{}name\PYZus{}\PYZus{}}\PY{p}{)}
\PY{n+nb}{print}\PY{p}{(}\PY{n}{ISZERO}\PY{p}{(}\PY{n}{TWO}\PY{p}{)}\PY{o}{.}\PY{n+nv+vm}{\PYZus{}\PYZus{}name\PYZus{}\PYZus{}}\PY{p}{)}
\PY{n+nb}{print}\PY{p}{(}\PY{n}{ISZERO}\PY{p}{(}\PY{n}{THREE}\PY{p}{)}\PY{o}{.}\PY{n+nv+vm}{\PYZus{}\PYZus{}name\PYZus{}\PYZus{}}\PY{p}{)}
\end{Verbatim}
\end{tcolorbox}

    \begin{Verbatim}[commandchars=\\\{\}]
TRUE
FALSE
FALSE
FALSE
    \end{Verbatim}

    \hypertarget{ux9012ux5f52}{%
\subsubsection{递归}\label{ux9012ux5f52}}

假如我们要写一个阶乘函数

    \begin{tcolorbox}[breakable, size=fbox, boxrule=1pt, pad at break*=1mm,colback=cellbackground, colframe=cellborder]
\prompt{In}{incolor}{35}{\boxspacing}
\begin{Verbatim}[commandchars=\\\{\}]
\PY{k+kn}{from} \PY{n+nn}{math} \PY{k+kn}{import} \PY{n}{factorial}
\PY{k}{def} \PY{n+nf}{fact}\PY{p}{(}\PY{n}{n}\PY{p}{)}\PY{p}{:}
    \PY{k}{if} \PY{n}{n} \PY{o}{==} \PY{l+m+mi}{0}\PY{p}{:}
        \PY{k}{return} \PY{l+m+mi}{1}
    \PY{k}{else}\PY{p}{:}
        \PY{k}{return} \PY{n}{n}\PY{o}{*}\PY{n}{fact}\PY{p}{(}\PY{n}{n}\PY{o}{\PYZhy{}}\PY{l+m+mi}{1}\PY{p}{)}

\PY{n+nb}{print}\PY{p}{(}\PY{n}{fact}\PY{p}{(}\PY{l+m+mi}{3}\PY{p}{)}\PY{p}{)}
\PY{k}{assert} \PY{n}{fact}\PY{p}{(}\PY{l+m+mi}{9}\PY{p}{)} \PY{o}{==} \PY{n}{factorial}\PY{p}{(}\PY{l+m+mi}{9}\PY{p}{)}
\end{Verbatim}
\end{tcolorbox}

    \begin{Verbatim}[commandchars=\\\{\}]
6
    \end{Verbatim}

    如果要写成函数形式,首先要考虑的是怎么实现if else这样的流程控制。

如果\texttt{N==ZERO},那么\texttt{ISZERO(N)}就会等价于\texttt{TRUE},而\texttt{TRUE(ONE,n*f(n-1))}会返回\texttt{ONE}。

因此 \texttt{ISZERO(N)(ONE,n*f(n-1))} 就完成了流程控制。

    \begin{tcolorbox}[breakable, size=fbox, boxrule=1pt, pad at break*=1mm,colback=cellbackground, colframe=cellborder]
\prompt{In}{incolor}{36}{\boxspacing}
\begin{Verbatim}[commandchars=\\\{\}]
\PY{k}{def} \PY{n+nf}{FACT}\PY{p}{(}\PY{n}{N}\PY{p}{)}\PY{p}{:}
    \PY{k}{return} \PY{n}{ISZERO}\PY{p}{(}\PY{n}{N}\PY{p}{)}\PY{p}{(}\PY{n}{ONE}\PY{p}{,}\PY{n}{MULTIPLY}\PY{p}{(}\PY{n}{N}\PY{p}{,}\PY{n}{FACT}\PY{p}{(}\PY{n}{SUBTRACT}\PY{p}{(}\PY{n}{N}\PY{p}{,}\PY{n}{ONE}\PY{p}{)}\PY{p}{)}\PY{p}{)}\PY{p}{)}

\PY{k}{try}\PY{p}{:}
    \PY{n}{show}\PY{p}{(}\PY{n}{FACT}\PY{p}{(}\PY{n}{THREE}\PY{p}{)}\PY{p}{)}
\PY{k}{except} \PY{n+ne}{Exception} \PY{k}{as} \PY{n}{e}\PY{p}{:}
    \PY{n+nb}{print}\PY{p}{(}\PY{n+nb}{repr}\PY{p}{(}\PY{n}{e}\PY{p}{)}\PY{p}{)}
\end{Verbatim}
\end{tcolorbox}

    \begin{Verbatim}[commandchars=\\\{\}]
RecursionError('maximum recursion depth exceeded')
    \end{Verbatim}

    但在实际运行时,可以看到我们的函数出现了无限循环的问题,这是因为Python并不是惰性求值(lazy
evaluation)的,它会先计算参数,再把参数的值代入函数

    \begin{tcolorbox}[breakable, size=fbox, boxrule=1pt, pad at break*=1mm,colback=cellbackground, colframe=cellborder]
\prompt{In}{incolor}{37}{\boxspacing}
\begin{Verbatim}[commandchars=\\\{\}]
\PY{k}{def} \PY{n+nf}{f}\PY{p}{(}\PY{n}{a}\PY{p}{,}\PY{n}{b}\PY{p}{)}\PY{p}{:}
    \PY{k}{return} \PY{n}{a}

\PY{k}{try}\PY{p}{:}
    \PY{n}{f}\PY{p}{(}\PY{l+m+mi}{1}\PY{p}{,}\PY{l+m+mi}{1}\PY{o}{/}\PY{l+m+mi}{0}\PY{p}{)}
\PY{k}{except} \PY{n+ne}{Exception} \PY{k}{as} \PY{n}{e}\PY{p}{:}
    \PY{n+nb}{print}\PY{p}{(}\PY{n+nb}{repr}\PY{p}{(}\PY{n}{e}\PY{p}{)}\PY{p}{)}
\end{Verbatim}
\end{tcolorbox}

    \begin{Verbatim}[commandchars=\\\{\}]
ZeroDivisionError('division by zero')
    \end{Verbatim}

    在上面这个例子中,虽然我们并不需要参数b,但Python仍会先计算b=1/0,此时就返回了错误。

避免这种情况出现的方法就是,不直接传入参数,而是传入返回参数的函数:

    \begin{tcolorbox}[breakable, size=fbox, boxrule=1pt, pad at break*=1mm,colback=cellbackground, colframe=cellborder]
\prompt{In}{incolor}{38}{\boxspacing}
\begin{Verbatim}[commandchars=\\\{\}]
\PY{k}{def} \PY{n+nf}{f}\PY{p}{(}\PY{n}{a}\PY{p}{,}\PY{n}{b}\PY{p}{)}\PY{p}{:}
    \PY{k}{return} \PY{n}{a}\PY{p}{(}\PY{p}{)}

\PY{n}{f}\PY{p}{(}\PY{k}{lambda}\PY{p}{:} \PY{l+m+mi}{1}\PY{p}{,} \PY{k}{lambda}\PY{p}{:} \PY{l+m+mi}{1}\PY{o}{/}\PY{l+m+mi}{0}\PY{p}{)}
\end{Verbatim}
\end{tcolorbox}

            \begin{tcolorbox}[breakable, size=fbox, boxrule=.5pt, pad at break*=1mm, opacityfill=0]
\prompt{Out}{outcolor}{38}{\boxspacing}
\begin{Verbatim}[commandchars=\\\{\}]
1
\end{Verbatim}
\end{tcolorbox}
        
    此时Python计算的步骤为:

\begin{enumerate}
\def\labelenumi{\arabic{enumi}.}
\tightlist
\item
  \texttt{a=lambda:\ 1}
\item
  \texttt{b=lambda:\ 1/0}
\item
  \texttt{a()=1}
\item
  \texttt{return\ 1}
\end{enumerate}

使用类似的方法对我们的原函数进行一些小改造,最终就能正常进行了

    \begin{tcolorbox}[breakable, size=fbox, boxrule=1pt, pad at break*=1mm,colback=cellbackground, colframe=cellborder]
\prompt{In}{incolor}{39}{\boxspacing}
\begin{Verbatim}[commandchars=\\\{\}]
\PY{n}{LAZY\PYZus{}TRUE} \PY{o}{=} \PY{k}{lambda} \PY{n}{x}\PY{p}{,}\PY{n}{y}\PY{p}{:} \PY{n}{x}\PY{p}{(}\PY{p}{)}
\PY{n}{LAZY\PYZus{}FALSE} \PY{o}{=} \PY{k}{lambda} \PY{n}{x}\PY{p}{,}\PY{n}{y}\PY{p}{:} \PY{n}{y}\PY{p}{(}\PY{p}{)}
\PY{n}{LAZY\PYZus{}ISZERO} \PY{o}{=} \PY{k}{lambda} \PY{n}{N}\PY{p}{:} \PY{n}{N}\PY{p}{(}\PY{k}{lambda} \PY{n}{\PYZus{}}\PY{p}{:} \PY{n}{LAZY\PYZus{}FALSE}\PY{p}{)}\PY{p}{(}\PY{n}{LAZY\PYZus{}TRUE}\PY{p}{)}

\PY{k}{def} \PY{n+nf}{FACT}\PY{p}{(}\PY{n}{N}\PY{p}{)}\PY{p}{:}
    \PY{k}{return} \PY{n}{LAZY\PYZus{}ISZERO}\PY{p}{(}\PY{n}{N}\PY{p}{)}\PY{p}{(}\PY{k}{lambda}\PY{p}{:} \PY{n}{ONE}\PY{p}{,}\PY{k}{lambda}\PY{p}{:} \PY{n}{MULTIPLY}\PY{p}{(}\PY{n}{N}\PY{p}{,}\PY{n}{FACT}\PY{p}{(}\PY{n}{SUBTRACT}\PY{p}{(}\PY{n}{N}\PY{p}{,}\PY{n}{ONE}\PY{p}{)}\PY{p}{)}\PY{p}{)}\PY{p}{)}

\PY{n}{show}\PY{p}{(}\PY{n}{FACT}\PY{p}{(}\PY{n}{THREE}\PY{p}{)}\PY{p}{)}
\end{Verbatim}
\end{tcolorbox}

    \begin{Verbatim}[commandchars=\\\{\}]
6
    \end{Verbatim}

    \hypertarget{y-combinator}{%
\subsection{Y combinator}\label{y-combinator}}

如果我告诉你Y combinator的定义如下 \[
Y = \lambda f . ( \lambda x. f( x (x) ) ) ( \lambda x. f( x (x) ) )
\]
那你肯定就开始怀疑下面的内容是否能看懂,别担心,其实很简单,忘记上面这个公式继续往下看吧。

\hypertarget{ux4f9dux7136ux662fux9636ux4e58ux51fdux6570}{%
\subsubsection{依然是阶乘函数}\label{ux4f9dux7136ux662fux9636ux4e58ux51fdux6570}}

在上面的例子中,我们是这样定义阶乘函数的:

    \begin{tcolorbox}[breakable, size=fbox, boxrule=1pt, pad at break*=1mm,colback=cellbackground, colframe=cellborder]
\prompt{In}{incolor}{40}{\boxspacing}
\begin{Verbatim}[commandchars=\\\{\}]
\PY{k}{def} \PY{n+nf}{fact}\PY{p}{(}\PY{n}{n}\PY{p}{)}\PY{p}{:}
    \PY{k}{if} \PY{n}{n} \PY{o}{==} \PY{l+m+mi}{0}\PY{p}{:}
        \PY{k}{return} \PY{l+m+mi}{1}
    \PY{k}{else}\PY{p}{:}
        \PY{k}{return} \PY{n}{n}\PY{o}{*}\PY{n}{fact}\PY{p}{(}\PY{n}{n}\PY{o}{\PYZhy{}}\PY{l+m+mi}{1}\PY{p}{)}

\PY{n}{fact}\PY{p}{(}\PY{l+m+mi}{3}\PY{p}{)}
\end{Verbatim}
\end{tcolorbox}

            \begin{tcolorbox}[breakable, size=fbox, boxrule=.5pt, pad at break*=1mm, opacityfill=0]
\prompt{Out}{outcolor}{40}{\boxspacing}
\begin{Verbatim}[commandchars=\\\{\}]
6
\end{Verbatim}
\end{tcolorbox}
        
    对于熟练掌握递归的人来说,这样貌似是非常理所当然的,但实际想一想,多少有点不可思议,因为我们在定义\texttt{fact}函数的过程中,居然用到了\texttt{fact}本身!

熟悉数理逻辑的人都会对自指(self-reference)满怀敬畏之心: -
罗素使用了自指,引发了第三次数学危机 -
哥德尔使用了自指,得出了哥德尔不完备定理 -
图灵使用了自指,提出了停机测试悖论

为了更好地理解递归,我们现在要避免在完整定义\texttt{fact}之前调用\texttt{fact}。

那么我们将函数内部的\texttt{fact}替换成\texttt{f},这里的\texttt{f}就是我们需要的阶乘函数:

\begin{Shaded}
\begin{Highlighting}[]
\KeywordTok{def}\NormalTok{ fact(n):}
    \ControlFlowTok{if}\NormalTok{ n }\OperatorTok{==} \DecValTok{0}\NormalTok{:}
        \ControlFlowTok{return} \DecValTok{1}
    \ControlFlowTok{else}\NormalTok{:}
        \ControlFlowTok{return}\NormalTok{ n}\OperatorTok{*}\NormalTok{f(n}\OperatorTok{{-}}\DecValTok{1}\NormalTok{)}
\end{Highlighting}
\end{Shaded}

但乍一看,这个\texttt{f}是凭空出现的,因此为了提供\texttt{f},我们给\texttt{fact}再添加一个参数:

    \begin{tcolorbox}[breakable, size=fbox, boxrule=1pt, pad at break*=1mm,colback=cellbackground, colframe=cellborder]
\prompt{In}{incolor}{41}{\boxspacing}
\begin{Verbatim}[commandchars=\\\{\}]
\PY{k}{def} \PY{n+nf}{fact}\PY{p}{(}\PY{n}{f}\PY{p}{,}\PY{n}{n}\PY{p}{)}\PY{p}{:}
    \PY{k}{if} \PY{n}{n} \PY{o}{==} \PY{l+m+mi}{0}\PY{p}{:}
        \PY{k}{return} \PY{l+m+mi}{1}
    \PY{k}{else}\PY{p}{:}
        \PY{k}{return} \PY{n}{n}\PY{o}{*}\PY{n}{f}\PY{p}{(}\PY{n}{n}\PY{o}{\PYZhy{}}\PY{l+m+mi}{1}\PY{p}{)}
\end{Verbatim}
\end{tcolorbox}

    这样的定义是可行的吗?

如果我们调用\texttt{fact(3)}肯定不对,因为\texttt{fact}有两个参数要输入。

调用\texttt{fact(f,3)}也不对,\texttt{f}还是没有定义,既然\texttt{f}就是\texttt{fact},那么我们尝试调用\texttt{fact(fact,3)}

    \begin{tcolorbox}[breakable, size=fbox, boxrule=1pt, pad at break*=1mm,colback=cellbackground, colframe=cellborder]
\prompt{In}{incolor}{42}{\boxspacing}
\begin{Verbatim}[commandchars=\\\{\}]
\PY{n}{fact}\PY{p}{(}\PY{n}{fact}\PY{p}{,}\PY{l+m+mi}{3}\PY{p}{)}
\end{Verbatim}
\end{tcolorbox}

    \begin{Verbatim}[commandchars=\\\{\}, frame=single, framerule=2mm, rulecolor=\color{outerrorbackground}]
\textcolor{ansi-red}{---------------------------------------------------------------------------}
\textcolor{ansi-red}{TypeError}                                 Traceback (most recent call last)
\textcolor{ansi-green}{/var/folders/1y/8ypw\_bc55x5d69n0rnzpwxjr0000gn/T/ipykernel\_25288/2401241230.py} in \textcolor{ansi-cyan}{<module>}
\textcolor{ansi-green}{----> 1}\textcolor{ansi-red}{ }fact\textcolor{ansi-blue}{(}fact\textcolor{ansi-blue}{,}\textcolor{ansi-cyan}{3}\textcolor{ansi-blue}{)}

\textcolor{ansi-green}{/var/folders/1y/8ypw\_bc55x5d69n0rnzpwxjr0000gn/T/ipykernel\_25288/901191219.py} in \textcolor{ansi-cyan}{fact}\textcolor{ansi-blue}{(f, n)}
\textcolor{ansi-green-intense}{\textbf{      3}}         \textcolor{ansi-green}{return} \textcolor{ansi-cyan}{1}
\textcolor{ansi-green-intense}{\textbf{      4}}     \textcolor{ansi-green}{else}\textcolor{ansi-blue}{:}
\textcolor{ansi-green}{----> 5}\textcolor{ansi-red}{         }\textcolor{ansi-green}{return} n\textcolor{ansi-blue}{*}f\textcolor{ansi-blue}{(}n\textcolor{ansi-blue}{-}\textcolor{ansi-cyan}{1}\textcolor{ansi-blue}{)}

\textcolor{ansi-red}{TypeError}: fact() missing 1 required positional argument: 'n'
    \end{Verbatim}

    可以看到我们的第5行中,\texttt{f(n-1)}出现了问题,因为此时其实是\texttt{fact(n-1)},而\texttt{fact}是需要两个参数的,因此我们将其改为\texttt{f(f,n-1)}

    \begin{tcolorbox}[breakable, size=fbox, boxrule=1pt, pad at break*=1mm,colback=cellbackground, colframe=cellborder]
\prompt{In}{incolor}{43}{\boxspacing}
\begin{Verbatim}[commandchars=\\\{\}]
\PY{k}{def} \PY{n+nf}{fact}\PY{p}{(}\PY{n}{f}\PY{p}{,}\PY{n}{n}\PY{p}{)}\PY{p}{:}
    \PY{k}{if} \PY{n}{n} \PY{o}{==} \PY{l+m+mi}{0}\PY{p}{:}
        \PY{k}{return} \PY{l+m+mi}{1}
    \PY{k}{else}\PY{p}{:}
        \PY{k}{return} \PY{n}{n}\PY{o}{*}\PY{n}{f}\PY{p}{(}\PY{n}{f}\PY{p}{,}\PY{n}{n}\PY{o}{\PYZhy{}}\PY{l+m+mi}{1}\PY{p}{)}
\end{Verbatim}
\end{tcolorbox}

    \begin{tcolorbox}[breakable, size=fbox, boxrule=1pt, pad at break*=1mm,colback=cellbackground, colframe=cellborder]
\prompt{In}{incolor}{44}{\boxspacing}
\begin{Verbatim}[commandchars=\\\{\}]
\PY{n}{fact}\PY{p}{(}\PY{n}{fact}\PY{p}{,}\PY{l+m+mi}{3}\PY{p}{)}
\end{Verbatim}
\end{tcolorbox}

            \begin{tcolorbox}[breakable, size=fbox, boxrule=.5pt, pad at break*=1mm, opacityfill=0]
\prompt{Out}{outcolor}{44}{\boxspacing}
\begin{Verbatim}[commandchars=\\\{\}]
6
\end{Verbatim}
\end{tcolorbox}
        
    这回终于正确了,但其实我们想要的阶乘函数应该是\texttt{fact(3)=6},于是我们可以稍微修改一下:

    \begin{tcolorbox}[breakable, size=fbox, boxrule=1pt, pad at break*=1mm,colback=cellbackground, colframe=cellborder]
\prompt{In}{incolor}{45}{\boxspacing}
\begin{Verbatim}[commandchars=\\\{\}]
\PY{k}{def} \PY{n+nf}{F}\PY{p}{(}\PY{n}{f}\PY{p}{,}\PY{n}{n}\PY{p}{)}\PY{p}{:}
    \PY{k}{if} \PY{n}{n} \PY{o}{==} \PY{l+m+mi}{0}\PY{p}{:}
        \PY{k}{return} \PY{l+m+mi}{1}
    \PY{k}{else}\PY{p}{:}
        \PY{k}{return} \PY{n}{n}\PY{o}{*}\PY{n}{f}\PY{p}{(}\PY{n}{f}\PY{p}{,}\PY{n}{n}\PY{o}{\PYZhy{}}\PY{l+m+mi}{1}\PY{p}{)}
\PY{n}{fact} \PY{o}{=} \PY{k}{lambda} \PY{n}{n}\PY{p}{:} \PY{n}{F}\PY{p}{(}\PY{n}{F}\PY{p}{,}\PY{n}{n}\PY{p}{)}
\PY{n}{fact}\PY{p}{(}\PY{l+m+mi}{3}\PY{p}{)}
\end{Verbatim}
\end{tcolorbox}

            \begin{tcolorbox}[breakable, size=fbox, boxrule=.5pt, pad at break*=1mm, opacityfill=0]
\prompt{Out}{outcolor}{45}{\boxspacing}
\begin{Verbatim}[commandchars=\\\{\}]
6
\end{Verbatim}
\end{tcolorbox}
        
    将其柯里化得到

    \begin{tcolorbox}[breakable, size=fbox, boxrule=1pt, pad at break*=1mm,colback=cellbackground, colframe=cellborder]
\prompt{In}{incolor}{46}{\boxspacing}
\begin{Verbatim}[commandchars=\\\{\}]
\PY{n}{F} \PY{o}{=} \PY{k}{lambda} \PY{n}{f}\PY{p}{:} \PY{k}{lambda} \PY{n}{n}\PY{p}{:} \PY{l+m+mi}{1} \PY{k}{if} \PY{n}{n}\PY{o}{==}\PY{l+m+mi}{0} \PY{k}{else} \PY{n}{n}\PY{o}{*}\PY{n}{f}\PY{p}{(}\PY{n}{f}\PY{p}{)}\PY{p}{(}\PY{n}{n}\PY{o}{\PYZhy{}}\PY{l+m+mi}{1}\PY{p}{)}
\PY{n}{fact} \PY{o}{=} \PY{n}{F}\PY{p}{(}\PY{n}{F}\PY{p}{)}
\PY{n}{fact}\PY{p}{(}\PY{l+m+mi}{3}\PY{p}{)}
\PY{k}{assert} \PY{n}{fact}\PY{p}{(}\PY{l+m+mi}{9}\PY{p}{)} \PY{o}{==} \PY{n}{factorial}\PY{p}{(}\PY{l+m+mi}{9}\PY{p}{)}
\end{Verbatim}
\end{tcolorbox}

    让我们回到最初的定义:

    \begin{tcolorbox}[breakable, size=fbox, boxrule=1pt, pad at break*=1mm,colback=cellbackground, colframe=cellborder]
\prompt{In}{incolor}{47}{\boxspacing}
\begin{Verbatim}[commandchars=\\\{\}]
\PY{n}{fact} \PY{o}{=} \PY{p}{(}\PY{k}{lambda} \PY{n}{f}\PY{p}{:} \PY{k}{lambda} \PY{n}{n}\PY{p}{:} \PY{l+m+mi}{1} \PY{k}{if} \PY{n}{n}\PY{o}{==}\PY{l+m+mi}{0} \PY{k}{else} \PY{n}{n}\PY{o}{*}\PY{n}{f}\PY{p}{(}\PY{n}{n}\PY{o}{\PYZhy{}}\PY{l+m+mi}{1}\PY{p}{)}\PY{p}{)}\PY{p}{(}\PY{n}{fact}\PY{p}{)}
\PY{n}{fact}\PY{p}{(}\PY{l+m+mi}{4}\PY{p}{)}
\end{Verbatim}
\end{tcolorbox}

            \begin{tcolorbox}[breakable, size=fbox, boxrule=.5pt, pad at break*=1mm, opacityfill=0]
\prompt{Out}{outcolor}{47}{\boxspacing}
\begin{Verbatim}[commandchars=\\\{\}]
24
\end{Verbatim}
\end{tcolorbox}
        
    如果我们用\texttt{R}来代替\texttt{lambda\ f:\ lambda\ n:\ 1\ if\ n==0\ else\ n*f(n-1)},那么就有
\texttt{fact\ =\ R(fact)},\texttt{fact}就是\texttt{R}的一个不动点

    \begin{tcolorbox}[breakable, size=fbox, boxrule=1pt, pad at break*=1mm,colback=cellbackground, colframe=cellborder]
\prompt{In}{incolor}{48}{\boxspacing}
\begin{Verbatim}[commandchars=\\\{\}]
\PY{n}{R} \PY{o}{=} \PY{k}{lambda} \PY{n}{f}\PY{p}{:} \PY{k}{lambda} \PY{n}{n}\PY{p}{:} \PY{l+m+mi}{1} \PY{k}{if} \PY{n}{n}\PY{o}{==}\PY{l+m+mi}{0} \PY{k}{else} \PY{n}{n}\PY{o}{*}\PY{n}{f}\PY{p}{(}\PY{n}{n}\PY{o}{\PYZhy{}}\PY{l+m+mi}{1}\PY{p}{)}
\PY{n}{fact} \PY{o}{=} \PY{n}{R}\PY{p}{(}\PY{n}{fact}\PY{p}{)}
\PY{n}{fact}\PY{p}{(}\PY{l+m+mi}{4}\PY{p}{)}
\end{Verbatim}
\end{tcolorbox}

            \begin{tcolorbox}[breakable, size=fbox, boxrule=.5pt, pad at break*=1mm, opacityfill=0]
\prompt{Out}{outcolor}{48}{\boxspacing}
\begin{Verbatim}[commandchars=\\\{\}]
24
\end{Verbatim}
\end{tcolorbox}
        
    假设我们有一个函数\texttt{Y(R)},可以计算出\texttt{R}的不动点,即\texttt{Y(R)\ =\ R(Y(R))}

\begin{verbatim}
F    = lambda f: lambda n: 1 if n==0 else n*f(f)(n-1)
F(F) =           lambda n: 1 if n==0 else n*F(F)(n-1)
fact =           lambda n: 1 if n==0 else n*fact(n-1)
<==> fact = F(F)

fact = (lambda f: lambda n: 1 if n==0 else n*f(n-1))(fact)
R    = lambda f: lambda n: 1 if n==0 else n*f(n-1)
<==> fact = R(fact)

F(x) = lambda f: lambda n: 1 if n==0 else n*f(f)(n-1) (x)
     =           lambda n: 1 if n==0 else n*x(x)(n-1)
     = R(x(x))
<==> F = lambda x: R(x(x))

Y(R) = fact
     = R(fact)
     = R(F(F))
<==> Y(R) = R(F(F))
\end{verbatim}

通过这些恒等式,我们可以通过\texttt{R(F(F))}来构建\texttt{Y(R)=fact}

    \begin{tcolorbox}[breakable, size=fbox, boxrule=1pt, pad at break*=1mm,colback=cellbackground, colframe=cellborder]
\prompt{In}{incolor}{49}{\boxspacing}
\begin{Verbatim}[commandchars=\\\{\}]
\PY{n}{R} \PY{o}{=} \PY{k}{lambda} \PY{n}{f}\PY{p}{:} \PY{k}{lambda} \PY{n}{n}\PY{p}{:} \PY{l+m+mi}{1} \PY{k}{if} \PY{n}{n}\PY{o}{==}\PY{l+m+mi}{0} \PY{k}{else} \PY{n}{n}\PY{o}{*}\PY{n}{f}\PY{p}{(}\PY{n}{n}\PY{o}{\PYZhy{}}\PY{l+m+mi}{1}\PY{p}{)}
\PY{n}{F} \PY{o}{=} \PY{k}{lambda} \PY{n}{x}\PY{p}{:} \PY{n}{R}\PY{p}{(}\PY{n}{x}\PY{p}{(}\PY{n}{x}\PY{p}{)}\PY{p}{)}
\PY{n}{Y} \PY{o}{=} \PY{k}{lambda} \PY{n}{R}\PY{p}{:} \PY{n}{R}\PY{p}{(}\PY{n}{F}\PY{p}{(}\PY{n}{F}\PY{p}{)}\PY{p}{)}

\PY{k}{try}\PY{p}{:}
    \PY{n}{fact} \PY{o}{=} \PY{n}{Y}\PY{p}{(}\PY{n}{R}\PY{p}{)}
\PY{k}{except} \PY{n+ne}{Exception} \PY{k}{as} \PY{n}{e}\PY{p}{:}
    \PY{n+nb}{print}\PY{p}{(}\PY{n+nb}{repr}\PY{p}{(}\PY{n}{e}\PY{p}{)}\PY{p}{)}
\end{Verbatim}
\end{tcolorbox}

    \begin{Verbatim}[commandchars=\\\{\}]
RecursionError('maximum recursion depth exceeded')
    \end{Verbatim}

    依然是因为惰性求值的问题,我们把 \texttt{x(x)} 换成
\texttt{lambda\ z:\ x(x)(z)},延后计算 x 的值。

    \begin{tcolorbox}[breakable, size=fbox, boxrule=1pt, pad at break*=1mm,colback=cellbackground, colframe=cellborder]
\prompt{In}{incolor}{50}{\boxspacing}
\begin{Verbatim}[commandchars=\\\{\}]
\PY{n}{R} \PY{o}{=} \PY{k}{lambda} \PY{n}{f}\PY{p}{:} \PY{k}{lambda} \PY{n}{n}\PY{p}{:} \PY{l+m+mi}{1} \PY{k}{if} \PY{n}{n}\PY{o}{==}\PY{l+m+mi}{0} \PY{k}{else} \PY{n}{n}\PY{o}{*}\PY{n}{f}\PY{p}{(}\PY{n}{n}\PY{o}{\PYZhy{}}\PY{l+m+mi}{1}\PY{p}{)}
\PY{n}{F} \PY{o}{=} \PY{k}{lambda} \PY{n}{x}\PY{p}{:} \PY{n}{R}\PY{p}{(}\PY{k}{lambda} \PY{n}{z}\PY{p}{:} \PY{n}{x}\PY{p}{(}\PY{n}{x}\PY{p}{)}\PY{p}{(}\PY{n}{z}\PY{p}{)}\PY{p}{)}
\PY{n}{Y} \PY{o}{=} \PY{k}{lambda} \PY{n}{R}\PY{p}{:} \PY{n}{R}\PY{p}{(}\PY{n}{F}\PY{p}{(}\PY{n}{F}\PY{p}{)}\PY{p}{)}
\PY{n}{fact} \PY{o}{=} \PY{n}{Y}\PY{p}{(}\PY{n}{R}\PY{p}{)}

\PY{n}{fact}\PY{p}{(}\PY{l+m+mi}{4}\PY{p}{)}
\end{Verbatim}
\end{tcolorbox}

            \begin{tcolorbox}[breakable, size=fbox, boxrule=.5pt, pad at break*=1mm, opacityfill=0]
\prompt{Out}{outcolor}{50}{\boxspacing}
\begin{Verbatim}[commandchars=\\\{\}]
24
\end{Verbatim}
\end{tcolorbox}
        
    \hypertarget{ux4e00ux822cux5f62ux5f0f}{%
\subsubsection{一般形式}\label{ux4e00ux822cux5f62ux5f0f}}

对于一般的情况,我们怎么计算\(f\)的不动点呢?

首先定义一个 \[
F(x)=f(x(x))
\] 于是有 \(F(F)=f(F(F))\),然后如下定义即可 \[
Y(f) = F(F)
\]

证明: \[
\begin{aligned}
Y(f) &= F(F) \\
   &= f(F(F))\\
   &= f(Y(f))\\
\end{aligned}
\] 于是可知 \(Y(f)\) 是 \(f\) 的不动点。

    \begin{tcolorbox}[breakable, size=fbox, boxrule=1pt, pad at break*=1mm,colback=cellbackground, colframe=cellborder]
\prompt{In}{incolor}{51}{\boxspacing}
\begin{Verbatim}[commandchars=\\\{\}]
\PY{k}{def} \PY{n+nf}{Y}\PY{p}{(}\PY{n}{f}\PY{p}{)}\PY{p}{:}
    \PY{n}{F} \PY{o}{=} \PY{k}{lambda} \PY{n}{x}\PY{p}{:} \PY{n}{f}\PY{p}{(}\PY{k}{lambda} \PY{n}{z}\PY{p}{:} \PY{n}{x}\PY{p}{(}\PY{n}{x}\PY{p}{)}\PY{p}{(}\PY{n}{z}\PY{p}{)}\PY{p}{)}
    \PY{k}{return} \PY{n}{F}\PY{p}{(}\PY{n}{F}\PY{p}{)}
\end{Verbatim}
\end{tcolorbox}

    我们还可以试着计算一下斐波拉契数列:

    \begin{tcolorbox}[breakable, size=fbox, boxrule=1pt, pad at break*=1mm,colback=cellbackground, colframe=cellborder]
\prompt{In}{incolor}{52}{\boxspacing}
\begin{Verbatim}[commandchars=\\\{\}]
\PY{k}{def} \PY{n+nf}{Fib}\PY{p}{(}\PY{n}{n}\PY{p}{)}\PY{p}{:}
    \PY{k}{if} \PY{n}{n}\PY{o}{==}\PY{l+m+mi}{1} \PY{o+ow}{or} \PY{n}{n}\PY{o}{==}\PY{l+m+mi}{2}\PY{p}{:}
        \PY{k}{return} \PY{l+m+mi}{1}
    \PY{k}{else}\PY{p}{:}
        \PY{k}{return} \PY{n}{Fib}\PY{p}{(}\PY{n}{n}\PY{o}{\PYZhy{}}\PY{l+m+mi}{1}\PY{p}{)}\PY{o}{+}\PY{n}{Fib}\PY{p}{(}\PY{n}{n}\PY{o}{\PYZhy{}}\PY{l+m+mi}{2}\PY{p}{)}

\PY{k}{for} \PY{n}{i} \PY{o+ow}{in} \PY{n+nb}{range}\PY{p}{(}\PY{l+m+mi}{1}\PY{p}{,}\PY{l+m+mi}{10}\PY{p}{)}\PY{p}{:}
    \PY{n+nb}{print}\PY{p}{(}\PY{n}{Fib}\PY{p}{(}\PY{n}{i}\PY{p}{)}\PY{p}{)}
\end{Verbatim}
\end{tcolorbox}

    \begin{Verbatim}[commandchars=\\\{\}]
1
1
2
3
5
8
13
21
34
    \end{Verbatim}

    \begin{tcolorbox}[breakable, size=fbox, boxrule=1pt, pad at break*=1mm,colback=cellbackground, colframe=cellborder]
\prompt{In}{incolor}{53}{\boxspacing}
\begin{Verbatim}[commandchars=\\\{\}]
\PY{n}{R} \PY{o}{=} \PY{k}{lambda} \PY{n}{f}\PY{p}{:} \PY{k}{lambda} \PY{n}{n}\PY{p}{:} \PY{l+m+mi}{1} \PY{k}{if} \PY{n}{n}\PY{o}{==}\PY{l+m+mi}{1} \PY{o+ow}{or} \PY{n}{n}\PY{o}{==}\PY{l+m+mi}{2} \PY{k}{else} \PY{n}{f}\PY{p}{(}\PY{n}{n}\PY{o}{\PYZhy{}}\PY{l+m+mi}{1}\PY{p}{)}\PY{o}{+}\PY{n}{f}\PY{p}{(}\PY{n}{n}\PY{o}{\PYZhy{}}\PY{l+m+mi}{2}\PY{p}{)}
\PY{n}{fib} \PY{o}{=} \PY{n}{Y}\PY{p}{(}\PY{n}{R}\PY{p}{)}
\PY{k}{for} \PY{n}{i} \PY{o+ow}{in} \PY{n+nb}{range}\PY{p}{(}\PY{l+m+mi}{1}\PY{p}{,}\PY{l+m+mi}{10}\PY{p}{)}\PY{p}{:}
    \PY{n+nb}{print}\PY{p}{(}\PY{n}{fib}\PY{p}{(}\PY{n}{i}\PY{p}{)}\PY{p}{)}
\end{Verbatim}
\end{tcolorbox}

    \begin{Verbatim}[commandchars=\\\{\}]
1
1
2
3
5
8
13
21
34
    \end{Verbatim}

    \hypertarget{reference}{%
\subsection{Reference}\label{reference}}

\begin{enumerate}
\def\labelenumi{\arabic{enumi}.}
\tightlist
\item
  \href{http://mindhacks.cn/2006/10/15/cantor-godel-turing-an-eternal-golden-diagonal/}{康托尔、哥德尔、图灵------永恒的金色对角线(rev\#2)
  -- 刘未鹏 \textbar{} Mind Hacks}
\item
  \href{http://www.inf.fu-berlin.de/lehre/WS03/alpi/lambda.pdf}{A
  Tutorial Introduction to the Lambda Calculus}
\item
  \href{https://blog.ruo-chen.wang/2021/04/lambda-calculus-from-the-ground-up.html}{从零开始的
  λ 演算 \textbar{} weirane's blog}
\item
  \href{https://www.youtube.com/watch?v=pkCLMl0e_0k\&ab_channel=PyCon2019}{David
  Beazley - Lambda Calculus from the Ground Up - PyCon 2019 - YouTube}
\item
  \href{https://github.com/orsinium-labs/python-lambda-calculus}{GitHub
  - orsinium-labs/python-lambda-calculus: Lambda Calculus things
  implemented on Python}
\end{enumerate}


    % Add a bibliography block to the postdoc
    
    
    
\end{document}
